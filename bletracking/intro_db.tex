%and !TEX root = paper.tex
\section{Introduction}
\label{sec:intro}

Passively listening to wireless tranmission from portable electronic devices like  
smartphones, smartwatches can reveal location of the user. An attacker could leverage these 
wireless transmission to track a user without their consent and pose significant privacy concern.  
% introduce a
%significant privacy threat: an attacker can track a user's
%location~\cite{locpriv-pervasive}.  For instance, a malicious application can
%actively share a device's identity and GPS coordinates with a third party.
This is feasible because many transmissions from portable devices have limited
range; receiving a transmission implies the transmitting device must be nearby.
Fortunately, these attacks are difficult to execute in practice: protocols on
portable devices like WiFi only communicate intermittently in
order to preserve battery life~\cite{}. Therefore, devices can only be tracked 
opportunistically with WiFi: namely, when the device happens to communicate. In addition,
these wireless protocols often hide the identity of the transmitter by randomizing
MAC addresses and encrypting payloads.
%
%In addition, many devices do not reveal their identity in their transmissions. 

%the device transmitting them (e.g., MAC address).

%can not be tracked continuously, 

Recently though, devices manufacturers have integrated Bluetooth Low Energy
(BLE) transmitters into many popular portable devices, such as iPhones,
FitBits, and Bose headphones. Unfortunately, BLE has the potential of being
significantly easier for an attacker to track the user. One of BLE's primary features
is continuous transmission of ``advertisement'' messages.  Advertisements
indicate the presence of the device to nearby devices to help locate a missing
device, and to provide seamless handoff between devices (e.g., Apple's
continuity protocol). BLE advertisements require negligible energy to transmit,
so they can be sent frequently (e.g., once a second) and continuously (i.e.,
the entire time a device is powered on)~\cite{}. Even contact tracing app relies on the BLE advertisements as they provide continous transmission mechanism on smartphones. 


Fortunately, the BLE protocol includes privacy protections that hide the identity of
the device in advertisements. Namely, the BLE protocol randomizes the MAC address of the
device every 10--15~minutes. Several attacks have been attempted on MAC addresses, which
later have been fixed with small design changes and required continous tracking of the
devices~\cite{Iphonetracking_becker,MACRandomizationfail_Martin}. \todo{describe one 
such example} Thus, the simple protection of MAC randomization still has prevented 
attackers from using BLE's continuous transmissions to track devices. 


% no physical layer signature exists for ble... it's simple and easy modulation 
% no rf finger print that can seperate 

In this work, we present [name], that demonstrate a new attack on BLE devices that can uniquely
identify a device based on a physical-layer fingerprint in each user device 
ble transmitter's hardware. 
%
[name] attack uses a low-cost, battery powered SDR, which one can carry around in their pocket to 
passively sense the \todo{entire?} 2.4 GHz channel, so that even frequency hopping can 
be easily tracked. 
[name] attack is simple, an attacker carrying pocket-SDR needs to be in proximity of a target device once 
to builds a model for the target device and then it can track the target device in the field, and 
requires no continuous tracking of the device. 
%
Specifically, [name] builds a physical-layer fingerprinting model for the users BLE 
transmitter, which it leverages to track the device in the field with high-fidelity.
%
[name] also establishes a field experiment methodology with defining clear goal of minimizing the
 false positive and false negative, which would serve as a benchmark for all future work. \todo{improve previous sentence}
%
Futhermore, since [name]'s attack models the transceiver hardware, one need to change the hardware to mitgate the 
attack, there is no other way for one to get around. 
We provide discussion on stratigies to mitigate such attacks in the future.

%
Radio Frequency (RF) fingerprinting techniques extract hardware 
impairments of the transmitter chain from the physical layer signal. 
Such techniques have been studied as a defense mechanism (like authentication, 
rogue device identification) for over a decade. Furthermore, these physical-layer
fingerprinting mostly have been demonstrated on protocols that use dense high-speed 
modulation techniques, such as WiFi~\cite{}.
%
A natural question, if there are unique physical-layer fingerprints for BLE transceivers,
 which can be used for RF fingerprinting. 
%
To understand the above, we make a key observation that most of Smartphone 
and Smartwatches have a combo chip WiFi+BLE chip, i.e. a single chip which
does both BLE and WiFi, saving chip-area by sharing the core transcevier 
compoenents except for the modem. 
%
With the two key observations, 1) the WiFi chips have unique physical layer
 fingerprints 2) the BLE is deployed as WiFi-BLE combo chip. We hypothesis that 
 the unique features for WiFi can be generalized to BLE, to extract physical layer 
 RF fingerprints. We show with extensive experimentation with 121 devices that the carrier frequency 
 offset and IQ imperfections (IQ offset and IQ imbalance) and such feastures that are 
 known to be distinguishable for WiFi, do exist for BLE as well. 

If these features are extracted for BLE, could be used for hardware phsyical layer identification. 
However, the extraction of these physical-layer Rf-identification of a BLE device is much more 
challenging than WiFi.
%
Specifically, BLE signals are shorter in time, narrow bandwidth, less dense, and more naive than WiFi signals; therefore, they provide significantly less information for an attacker to fingerprint the physical layer characteristics of the transmitter.
%
In constrat, WiFi in particular, lends itself to an accurate estimation of transmitter's physical layer characteristics like CFO and IQ offset, as part of decoding process i.e., the preamble of WiFi has signal features like LTF\todo{full-form of LTF} and multiple subcarriers to enable extraction of the physical layer fingerprints.
In contrast, BLE has short preamble (the known bit sequence at the beginning of the
packet), low bitrates (1~Mbps), and simple
Frequency Shift Keying (FSK) modulation scheme give an attacker much less information to profile 
physical layer features of a transmitter (See Section [ref] for details). In fact, no prior work
 has demonstrated that it is possible to obtain accurate and identifiable estimates of hardware
  imperfections for BLE transmitters. Having a fine-grained and accurate hardware imperfection 
  estimation is of a significant importance when we want to be able to identify a particular 
  device among a huge set of other devices, potentially with hardware imperfections close to our device. 


We develop a novel technique that helps us to combat the simple nature of BLE signals (low bandwidth and FSK transmission) and provide accurate and fine-grained estimates of hardware imperfections of the transmitter. One of the main challenges is that the preamble which is the known part of the signal and in many RF fingerprinting schemes is used to estimate the fingerprints, is very short (8 microseconds) for BLE, which impedes the estimation of transmitter impairments. We cannot augment the bandwidth, but we potentially aim to mitigate short preable time-window. Our insight is that we can decode the BLE message, then reconstruct an ideal
version of the signal from the decoded bits, then we have a longer known signal
than the preamble to compare with to find imperfections. This technique is particularly more effective for BLE signals as their imperfections don't affect the decoding process unlike WiFi.

Next, extracing the CFO and IQ imbalance simultaneously with an FSK signal is challenging, as \todo{...}
We model this insight as an optimization problem over the whole packet by mathematically modeling the imperfect signal, and find the primary imperfections---CFO and IQ imperfections---using a gradient descent based non-convex optimization technique. We found that this method significantly reduces the standard deviation of our estimation and provides a 10x accuracy improvement compared to the exisitng methods for estimating these imperfections.

%Although this signal is not complex enough to estimate two different imperfections independently, we demonstrate it is possible to jointly estimate the two primary imperfections---CFO and IQ imbalance---using a gradient descent based non-convex optimization technique. We found that this provides a 10x accuracy improvement compared to independently estimating CFO and IQ imbalance.

% \todo{not sure what to do with the next part, suggestion is to delete:
% We also are the first to describe the impairments introduced by the low-power
% BLE architectures.
% %
% These transmitters save energy by using analog components (rather than digital
% signal processing) to generate the FSK signal.  Specifically, they use
% an analog oscillator (called a PLL) that can deviate its generated frequency by
% positive deviation ($+f_d$) to represent a one, and by a negative deviation
% ($-f_d$) to represent a zero.
% %
% We examine these frequency deviation patterns for a set of 20 devices that are
% the same make and model, and we observed that there are patterns in the
% histogram of the frequency deviation, namely how the device represents a one
% and a zero.
% %
% Therefore, we demonstrate that one can fingerprint the unique transmitter
% characteristics of low-power devices by using a histogram of these frequency
% deviations.
% }

%profile the hardware impairments of the loop filter, we can characterize all
%kind of BLE devices. Furthermore, the loop filter are necessity for an BLE
%chip, hence it enables us to achieve universal RF signature. A natural
%question, is how can we learn the loop filter characteristics. O We build a new RF signature based on this observation, specifically,


%no more uses the IQ architecture for RF transmission, instead use digital PLL
%with are fed directly with a digital signal. A consequence of these
%architectures therefore is that we can no more have fixed hardware impairments
%as carrier frequency offset and IQ imbalance. The new chipsets have a large
%market where they are using for fitness trackers to reduce the radio power
%consumptions. This requires us to identify a new signature to classify these
%devices. 
%introducing
%a new technique that profiles the 
%
%Finally, we demonstrate that awe can combine 

%large packet size and mutli-carrier modulation, which require
%estimation of the hardware impairments like CFO and IQ imbalance as part of
%standard decoding scheme. A natural question is how can we extract these
%parameters (CFO and IQ imbalance) from BLE transmission as accurate as WiFi
%transmissions.
%
%Most techniques employ signal processing algorithms as well as classical
%machine learning and new deep learning architectures to extract and classify
%hardware impairments.

%Furthermore, we make an interesting observation that most smartphones use a
%combo-chip for BLE and WiFi devices, therefore if such signatures can be
%detected using BLE, we would achieve and ubiquitous~\todo{use different word}
%attack scheme which can track user everywhere and continuously. 
%%
%We profile multiple different BLE chips, and learn that there are multiple chip
%design architectures. We build an universal RF signature scheme, which allows
%to create a unique signature across different chip architecture and build a
%robust attack. 

%By leveraging the non-linear optimization using gradient descent, we can
%identify devices which use combo-chip for WiFi and BLE where they share similar
%hardware architecture for RF front-end, and therefore can use RF signature
%based on CFO and IQ imbalance. However, turns out that in order to further
%reduce the power consumption there are new architecture for BLE devices which

\todo{four principle of writing, what is the problem? why is it hard? naive solutions and your amazing solution?}
Finally, extraction of the hardware impairments if it is sufficient for an attack to be sucessful, requires
an extensive study. We conduct extensive experiments to demonstrate the feasibility of our 
RF fingerprinting attack in different situations. We first devise a methodology that is accurate 
enough to distinguish devices from the same make and model with 95-100 percent accuracy in a 
controlled lab environment at different SNR levels. Then we collect packets from a huge set of 
devices obserevd in the field at different places such as coffee shops and library to demonstrate
 that our method still works in an uncontrolled setting and to gain insights about how much 
 distinguishable the actual devices in-use are. We analyze the data and find out the hardware 
 imperfection features of these devices collected in a completely uncontrolled environment, 
 are distinguishable with an average False Positive Rate (FPR) of 1.21 percent and False Negative 
 Rate (FNR) of 2.53 percent. 

 However, we also demonstrate that not all devices are as distinguishable as others and the ability to identify them depends on how distinguishable or common their hardware imperfection features are. We then profile the imperfections of a set of 17 target devices and try to idenftify them in a new place a few days later. We show that we can identify these targets with FPR of 3.5 percent and FNR of 3.21 percent. Finally, we deploy an attack scenario study in which we profile a volunteer individual and track their presnece at their home over time. The person moves inside and outside of the home and we are succesfully able to detect their presence over time. This experiments demonstrates a realistic scenario in which RF fingerprinting attack can threaten the privacy of the person. Moreover, we experiment the effects of environmental conditions such as temperature on the robustness of RF fingerprints to evaluate practicality and limitations of RF fingerprinting in the real world. To the best of our knowledge, this is the first paper that provides evidence and evaluations on practicality of RF fingerprinting in the real world setting and also conducts experiments to demonstrate the limitations.

\if 0 % EXTRA TEXT {{{

In summary, the primary contributions of this work are as follows:

\begin{enumerate}
%
\item We discovered that many popular devices are frequently transmitting BLE
	advertisements (hundreds per minute) and how an attacker can missuse this fact to endanger the user's privacy.
%
\item We proposed that the same hardware imperfections that are known to be ditingushable for WiFi can be used to identify BLE transmitters. We also describe how low-speed short BLE transmissions are more difficult to
	fingerprint than other high-speed protocols such as WiFi.
%
\item We introduce a new RF fingerpritng technique that can robustly
	fingerprint BLE transmitter architectures. This method enables obtaining fine-grained and identifiable estimates of hardware imperfections for BLE/Bluetooth for the first time.
%
\item We collect signals from a huge set of random devices in the field and analyze them to evaluate their identifiability to gain insights on how ditinguishable actual devices in the field are. 
%
\item We deploy and evaluate our RF fingerprinting attack in a typical scenario that this attack might be done and demonstrate we can successfully track the presence of a person in a particular location over time.
% \item We design and deploy a prototype of the BLE fingerprinting attack, and
% 	through extensive measurements in the lab and in the real world we demonstrate
% 		that it can robustly fingerprint devices of the same make and model, and can differentiate
% 		devices that are operating in the wild. 
%
\item We evaluate the limitations and capabilities of RF fingerprinting in real-world settings and conditions.
%
\end{enumerate}



We demonstrate that these devices transmit frequently enough that the attack
can be performed in a short period of time, even when a user is quickly passing
by.
%
An attacker can track a device across multiple locations and over multiple
days.
%
We also demonstrate it is feasible to track many kinds of popular portable
devices from different vendors (e.g., the iPhone, FitBit).




, which acts
as fingerprints of the radio, in order to identify and classify the devices.

During the past 15 years, there have been lots of
effort in designing algorithms and protocols for building authentication and
identification systems based on physical-layer fingerprinting, especially for
WiFi technology.

RF fingerprinting has become explored for high-power devices, and shown
interesting possible solution for applications such as authorized access in
WiFi access points (AP).

, and multiple
locations, office buildings. Furthermore, we show that BLE transponders in
smart phones, smart watches and fitness trackers are frequent trying to
synchronize and advertise themselves, even with no traffic in the envrionment
or not other device to connect with, thereby making our attack robuts

Therefore, they can be used as unique signatures
to identify different devices.  Moreover, as they are caused by minor
variations in analog hardware components, they can be  even used to identify
devices from the same type and manufacturer.  In this work, we present an
attack for the first time, 

RF fingerprinting refers to the techniques which take advantage of hardware
imperfections produced during manufacturing and extract unique features related
to hardware component in the RF chain. As RF fingerprints are embedded in the
nature of the hardware, it is hard to hide or mimic these fingerprints. 



providing location
privacy through, personal device platforms such as Apple iOS, Android, and
FitBit have built in location privacy controls.  They also have worked to
protect users when privacy issues have been revealed, such as when GPS location
was inadvertently being 



portable devices by stalkers to
follow spouses~\cite{ipv}.  Location privacy has become a primary concern for
personal device platforms



protocols including BLE relies
on randomization of their identification, for example, MAC address
randomization. BLE protocol suggests changing a device's MAC address to a
randomly generated address after a while so that attackers lose the victim
device. A study shows that some devices were programmed with improper
randomization for BLE protocol and ended up sending wireless frames with the
true global address~\cite{MACRandomizationfail_Martin} resulting in
identification. Although such imperfection has been fixed in software in later
release of the operating system,

privacy consequences. Imagine
one carrying around his smart-phone which is frequently transmitting BLE
beacons is equivalent to announcing your presence to a malicious attacker.
Detecting presence of a BLE transmission from Smart phones combined with side
information from the physical world would end up in detecting the presence and
tracking a person daily activities.



Specifically, the BLE
(Bluetooth Low Energy) protocol has become the gold-standard for advertising,
beacons and interactions with physical surroundings, due to its wide
penetration in smart phones and low power profile of transmissions.  Specifically smart devices transmits 100x more BLE packets
compared to WiFi packets and are perodic. 



We are surrounded by devices which can help us interact with the surrounding,
specifically each of us carry smart devices like watches, phones and fitness
trackers, which interact with our environments. 

\fi
%}
