\section{Introduction}

The mobile devices we carry every day, such as smartphones and
smartwatches, increasingly function as wireless tracking
beacons. These devices continuously transmit short-range wireless
messages using the Bluetooth Low Energy (BLE) protocol.  These beacons
are used to indicate proximity to any passive receiver within range.
Popular examples of such beacons include the COVID-19 electronic
contact tracing provided on Apple and Google
Smartphones~\cite{9-millionca} as well as Apple's intrinsic Continuity
protocol, used for automated device hand-off and other proximity
features~\cite{applecontinuity}.

However, by their nature, BLE wireless tracking beacons have the
potential to introduce significant privacy risks. For example, an
adversary might stalk a user by placing BLE receivers near locations
they might visit and then record the presence of the user's
beacons~\cite{exposurefaqmanual,dp3t}.  To address these issues, common BLE proximity
applications cryptographically anonymize and periodically rotate the
identity of a mobile device in their beacons. For instance, BLE
devices periodically re-encrypt their MAC address, while still
allowing trusted devices to determine if these addresses match the
device's true MAC address~\cite{bluetoothprivacy}. Similarly, COVID-19 contact tracing
applications regularly rotate identifiers to ensure that receivers
cannot link beacons from the same device over time~\cite{exposurenotificationmanual}.

While these mechanisms can foreclose the use of beacon content as a
stable identifier, attackers can bypass these countermeasures by
fingerprinting the device at a lower layer. Specifically, prior work
has demonstrated that wireless transmitters have imperfections
introduced in manufacturing that produce a unique physical-layer
fingerprint for that device (e.g., Carrier Frequency Offset and I/Q
Offset). Physical-layer fingerprints can reliably differentiate many
kinds of wireless chipsets~\cite{rfidphysical_danev,Brik_radiometric,transientBT_Hall,suskitransient,tximperfections_polak,femtocell_kennedy,adsb,subgrfid}, including a recent attempt to
distinguish 10,000 WiFi~\cite{kaushik10000wifi} chipsets.

However, no prior work has evaluated the
practicality of such physical-layer identification attacks in a real-world
environment. 
%
Indeed, prior to BLE tracking beacons, no mobile device wireless
protocol transmitted frequently enough---especially when idle---to make
such an attack feasible. 
%
In contrast, today it is common to find tens and hundreds of personal devices transmitting these BLE beacons at all public locations -- office buildings, public library, coffee shops and others.
%
For an attacker, even with a precise fingerprint, locating one target device in such public locations is a needle in a haystack problem --- we don't know the limitations to uniquely differentiating one BLE device in this "noise" of several other BLE devices.
%

In this chapter, we take an empirical measurement approach to understanding the practicality of this tracking threat. 
%
We develop a technique to estimate high precision fingerprints from BLE beacons.
%
We then perform BLE beacon data collection in lab on devices that we control, and also uncontrolled data collection of BLE beacons from mobile devices seen at a variety of public locations.
%
Using our fingerprint technique, we analyze these beacons from real-world devices to understand the scope of this privacy threat, and how likely is an attacker at being successful in locating their target. 
%
In particular the contributions of our work are as below:

\begin{enumerate}
\item Using lab-bench experiments on BLE devices we own, we identify four primary challenges to identifying BLE devices in the field: (1) BLE devices have a variety of chipsets that have different
hardware implementations, (2) applications can configure the BLE transmit
power level, resulting in some devices having lower SNR BLE transmissions,
(3) the temperature range that mobile devices encounter in the field
can introduce significant changes to physical-layer impairments, and (4) the low-cost
receivers that an attacker can use in the wild for RF fingerprinting may be significantly less accurate than the tools used in prior studies~\cite{Brik_radiometric}.
 
\item We perform an empirical study through a set of field experiments to evaluate how
significantly these challenges diminish an attacker's ability to identify
mobile devices in the field. We leverage the fact that BLE tracking beacons are
already used on many mobile devices to perform an uncontrolled field study
where we evaluate the feasibility of tracking BLE devices when they
are operating in public spaces where there are hundreds of other nearby devices.
 To the best of our knowledge, our work is
the first to evaluate the feasibility of an RF fingerprinting attack in
real-world scenarios.
\end{enumerate}

Through these empirical studies, we show that even when there are hundreds of devices we encountered in the field, it is still feasible to uniquely identify 
a specific mobile device by its physical-layer fingerprint. 
%
However, we also
observe that certain devices have similar fingerprints to others, and temperature
variations can change a device's metrics. 
%
Both of these issues can lead to significant confusion in distinguishing mobile devices.
%
In summary, we find that physical layer tracking of
BLE devices is indeed feasible, but it is only
reliable under limited conditions, and for specific devices with extremely
unique fingerprints, and when the target device has a relatively stable
temperature. 

