%the !TEX root = paper.tex
\begin{abstract}

Periodic transmissions of Bluetooth Low Energy (BLE) advertisements are
becoming popular in personal electronic devices.
%
These transmissions are continuous: many personal devices, including iPhones
and FitBits, transmit at least one advertisement per second, during the
entire time that the device is powered on.
%
The frequent transmission of these advertisements may make it possible for
attackers to track the location of personal devices by passively listening
for nearby advertisements messages.
%
This has not been feasible today, because BLE transmitters hide their
identity in these advertisements by randomizing their MAC address.

In this paper, we demonstrate the first physical-layer fingerprinting attack
	that can be used to identify BLE devices.
%
We describe why the low-bitrate, short, transmissions, and low-power
	transmitter designs of BLE make it particularly challenging to fingerprint,
	and why prior fingerprinting techniques do not work. 
%because BLE transmissions are short, low-bitrate
%
%Also, some BLE transmitters often use novel low-power transmitter
	%architectures that can not be fingerprinted with WiFi-like techniques.
We present two new fingerprinting techniques that can be combined to fingerprint all
	types of BLE transmitters.
%
Then we demonstrated these techniques can be combined with ML classification
	algorithms to identify devices maliciously in real-world settings.
%
We perform a controlled evaluation of this system with 40 devices of the same
	make and model in the lab, and using real-world evaluation in public settings
	(e.g., coffee shops) we demonstrate that targets can be identified in
	real-world environments where they are surrounded by hundreds of other BLE devices.

\end{abstract}














\begin{comment} 
	BLE devices use advertisements to "announce their presence to
the world", i.e., enable detection of peripherals by BLE central devices. While
this mechanism is low power consuming and convenient it raises potential privacy
concerns. BLE advertisements are sent unencrypted, and typically contain the MAC
address of the peripheral in the payload. An attacker can use this to track
devices, and in the case of personal consumer devices (smartphones, headphones,
smartwatches) even track an individual.

To avoid blatant misuse of advertisement packets by eavesdroppers, Bluetooth
standard provides the use of private addresses in advertising packets. These
addresses are randomly generated and keep changing according to a defined time
interval. This pegs back opportunistic tracking, as temporal advertisement
packet collection cannot be attributed to a specific device.

In this project, we attempt to bypass randomization by going low-level, i.e. the
Bluetooth PHY layer. We measure properties of BLE advertisements such as the
variation in carrier frequency offset of peripherals, to try and see if we can
uniquely associate them to a specific device, thereby bypassing the baseband
layer privacy protection (MAC randomization) 
\end{comment}
