%the !TEX root = paper.tex
\begin{abstract}

Bluetooth Low Energy (BLE) applications, such as digital contact tracing,
  have resulted in mobile devices continuously transmitting BLE beacons.
%
%Furthermore, these transmissions are becomining continuous: many personal
  %devices, including iPhones and FitBits, transmit at least one advertisement
  %per second, during the entire time that the device is powered on.
This frequent transmission of these beacons posses a significant location
  privacy concern: attackers can determine if a mobile device is nearby by
  passively listening for its beacons.
%
Although BLE devices randomize their MAC address, we show BLE radios have a uniquely
  identifiable physical-layer fingerprint.
  %
However, BLE is fundamentally difficult to fingerprint because of its low-bitrate, narrow-band,
  short, transmissions.
%  and present novel non-convex optimization based
%  techniques which can extract the rf-fingerprints with just the BLE
%  transmission. 
%
 In this paper, we demonstrate the first
  physical-layer fingerprinting attack that can be used to identify BLE devices.
  %We first present that BLE transmitters have uniquely
  %identifiable RF fingerprints.  
%why prior fingerprinting techniques do not work.  because BLE transmissions
  %are short, low-bitrate
%
%Also, some BLE transmitters often use novel low-power transmitter
  %architectures that can not be fingerprinted with WiFi-like tec
We also evaluate the effectiveness of this attack in real world settings with hundreds of BLE devices.
%
%\todo{fix last sentence with key results and evaluation} We perform a
  %controlled evaluation of this system with \todo{check}160 devices of the same
  %make and model in the lab, and using real-world evaluation in public settings
  %(e.g., coffee shops) we demonstrate that targets can be identified in
  %real-world environments where they are surrounded by hundreds of other BLE
 %devices.

\end{abstract}









