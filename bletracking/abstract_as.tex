\begin{abstract}

Mobile devices increasingly function as wireless tracking beacons. Using the
Bluetooth Low Energy (BLE) protocol, mobile devices such as smartphones and
smartwatches continuously transmit beacons to inform passive listeners about
device locations for applications such as digital contact tracing for
COVID-19, and even finding lost devices. These applications use cryptographic
anonymity that limit an adversary's ability to use these beacons to stalk a
user. However, attackers can bypass these defenses by fingerprinting the
unique physical-layer imperfections in the transmissions of specific
devices.

We empirically demonstrate that there are several key challenges that can limit
an attacker's ability to find a stable physical layer identifier to uniquely
identify mobile devices using BLE, including variations in the hardware
design of BLE chipsets, transmission power levels, differences in thermal
conditions, and limitations of inexpensive radios that can be widely deployed
to capture raw physical-layer signals. We evaluated how much each of these
factors limits accurate fingerprinting in a large-scale field study of
hundreds of uncontrolled BLE devices, revealing that physical-layer 
identification is a viable, although sometimes unreliable, way for an attacker to
track mobile devices.

\end{abstract}
