\section{Introduction}
\label{sec:hyperscanner:intro}
Classic Bluetooth ad-hoc links are distributed throughout large urban areas. 
%
Performing a comprehensive security audit of this "network" of non-Internet connected links requires us to perform wireless scanning over large geographic areas.
%
Wardriving wireless scanning can help solve this problem.
%
Indeed, as we have seen in the chapter on Bluetooth skimming, investigators were able to detect Bluetooth skimmers when performing wardriving based data collection.
%
In fact, they were even able to find skimmers when driving past the gas station.
%
Furthermore, this style of wardriving based enumeration of devices has been used in the past by companies such as Google to build databases of nearby WiFi access points.

Unfortunately conventional classic Bluetooth scanning is a slow process
%
Bluetooth scanning requires us to sequentially send device discovery (inquiry) packets on certain frequency channels and then wait for responses in a time slotted ALOHA manner. 
%
This process has to be done for 32 inquiry request/response channels spread across a bandwidth of 76 MHz in the 2.4 GHz ISM band.
%
Additionally, Bluetooth devices turn on their receivers for only short durations periodically (10 ms every 1.24 sec) to save power.
%
This requires that the scanning must be sequentially repeated several times across all the 32 channels, to ensure that every device sends a response.
%
Consequently, it takes at least 10.24 seconds to scan for every classic Bluetooth device within wireless range. 
%
In noisy environments, this duration can get even longer (upto 40.96 seconds).
%

The slow speed of single-channel Bluetooth scanning makes existing scanning tools like smartphones infeasible for wardriving data collection across entire urban areas. 
%
For example, at a driving speed of 50 mph, scanning for Bluetooth devices within a typical urban range of 100 meters requires us to finish the scan in less than 4 seconds.
%
Since current single-channel scanners need \~10 seconds to finish a scan, they will likely end up missing devices.
%
Consequently, we are left with an impossible choice -- we cover an entire metro area while continuously wardriving but miss Bluetooth devices, or we discover all Bluetooth devices in a smaller geographic area by doing stop-and-go scanning.
%
Either scenario significantly limits a comprehensive security audit of this distributed "network" of wireless links.
%

Modern software-defined radios can be used to design faster Bluetooth scanning tools, but they are limited by their cost and portability.
%
Indeed, a portable low-cost scanning tool lets us deploy hundreds of them to cover an entire urban area.
%
SDRs speed up the scan process by transmitting inquiry requests and receiving responses in parallel across multiple narrowband classic Bluetooth channels.
%
The received raw response signals can be backhauled to the host computer, where they can be decoded to retrieve the information about the devices being queried.
%
There also exist several low-cost portable SDR options such as PlutoSDR that provide an on-board processor and don't need a separate host computer.

%
Unfortunately, the bandwidth requirements of a parallel multi-channel Bluetooth scanning limit the choice of SDR hardware.
%
Classic Bluetooth scanning requires us to send multiple inquiry requests and receive inquiry responses across channels spread over an analog bandwidth of 76 MHz in the 2.4 GHz band. 
%
This further requires a network backhaul rate of $\sim$2.5 Gbps, which can only be supported using a 10G ethernet link.
%
This makes low-cost SDR options such as PlutoSDR unsuitable for a parallel multi-channel Bluetooth scanning application
%
Higher end SDRs can be used for this purpose, but they are often neither portable nor low-cost.

In this chapter, we present an initial exploration into the design of a low-cost multi-channel Bluetooth scanning tool, aimed at performing fast wireless audits across urban areas.
%
The key insight that helps us in the design of this tool is that Bluetooth uses separate request and response channels.
%
There are 32 inquiry request channels, and corresponding one-to-one mapped (but different) 32 response channels.
%
We can therefore send inquiry packets rapidly across all channels, and responses will arrive at defined independent response frequencies after a Bluetooth protocol defined time period (time slot).
%
The specific contributions of this work are:
\begin{enumerate}
    \item We present a Bluetooth-protocol compliant multi-channel scan algorithm to reduce the scan time to under 10 seconds.
    \item We present a novel technique for enabling low analog bandwidth SDRs (61.44 MHz) to access Bluetooth channels outside of their receive bandwidth
\end{enumerate}

The rest of the chapter is organized as follows: Section~\ref{sec:hyperscanner:bkgd} provides a background on the Bluetooth device discovery process, and prior work in speeding up Bluetooth scans. Section ~\ref{sec:hyperscanner:design} presents the design of the multi-channel scan algorithm, resolving hardware issues with multi-channel scan, and enabling out-of-band channel access for low-cost PlutoSDR. Section ~\ref{sec:hyperscanner:eval} provides initial testing results on the speed-up provided by the new scanning algorithm, and we conclude in Section~\ref{sec:hyperscanner:concl}.