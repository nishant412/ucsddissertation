\begin{dissertationabstract}
    Modern urban society relies upon the safe and secure operation of wireless communication links that are part of all modern electronic systems. 
    %
    From personal devices like smartphones to public infrastructure like grid equipment, wireless ad-hoc links using Bluetooth and WiFi radios, enable users to remotely access and passively monitor these electronic equipment, conveniently and safely at a distance.
    %
    
    Unfortunately, these non-Internet connected links have exposed a new threat vector.
    %
    Attackers can gain unauthorized access or remotely track our electronic devices using these links, from a distance with minimal risk of getting caught.
    %
    Securing these link requires us to perform a comprehensive auditing or identification using wireless scanning across urban areas.
    %
    This is complicated because these links are spread over entire metro areas, and are hard to identify as any one link is "hidden in the noise" of surrounding hundreds of links.
    %
    We need to empirically understand if auditing using wireless scanning is even a feasible approach in these urban area real-world situations.
    %
    Unfortunately, there is no prior empirical analysis that actually shows the feasibility of such scanning in real-world locations with tens of hundreds of other similar devices.
    %
    We also lack the tools to perform this wireless scanning over entire metro areas without missing devices.
    %

    In this dissertation, I perform large scale empirical studies and design tools to quantitatively and qualitatively analyze the feasibility of performing wireless link auditing in real-world urban environments.
%
I perform wireless auditing measurement studies to analyze the effectiveness of auditing both from an attacker and a defensive perspective.
%
I then present the design of a tool to perform practical faster scanning of wireless links for wardriving applications
%
I argue that we can distinguish individual target wireless links even in the presence of several other links around, even with the limited information revealed by wireless scans.


\end{dissertationabstract}