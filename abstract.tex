\begin{dissertationabstract}
    Securing networked computing devices requires us to comprehensively inventory all devices connected onto an organizational network. Such an inventory or identification helps us identify vulnerabilities in legitimate network connected devices, as well as any illegitimate devices on the network. However, there is a new class of networked computing devices based on short range peer-to-peer wireless links  that are extremely hard to inventory. These wireless links utilize commodity Bluetooth and WiFi radios, and present limited identifying information. In recent times, these wireless links have been widely utilized in critical urban infrastructure such as power grid equipment, street lights, traffic cameras, highway signs etc. While these short range links make it easier for maintenance personnel to run diagnostics and change configurations, they present several challenges to identification at an urban scale. They are inaccessible remotely through a centralized network, spread over large geographic areas and are practically invisible to scanning due to wireless noise of consumer Bluetooth and WiFi devices in the field. This challenge of inventory due to lack of visibility has not only opened up several potential attack surfaces in our critical infrastructure, but has also encouraged criminals to install some of their own wireless radios to gain unauthorized access to infrastructure.

    In this dissertation I tackle the problem of metropolitan scale inventory of urban infrastructure utilizing short range peer-to-peer wireless devices using wireless scan data collected from wardriving efforts. Firstly, I present a project in which I show that it is possible to uniquely identify illegal short range peer-to-peer wireless devices in fuel infrastructure based on Bluetooth scan information. Next, I present an ongoing project in which we analyze signal information to see if we can derive unique persistent identifiers for identification of such wireless radios in the field. Finally, I propose a project to perform large scale inventory of urban infrastructure based on scan data of short range wireless links they utilize, across multiple counties in the US. I discuss the technical and logistical challenges associated with such an effort, proposed ideas to tackle those challenges, and open research questions that I hope to answer.
    
\end{dissertationabstract}