%!TEX root = paper.tex
\section{Current Methods of Detection are Too Slow}
\label{sec:arizona}

Our goal in this paper is to demonstrate the necessity of using Bluetooth for the detection of fuel station card
skimmers. A primary motivation for this is that Bluetooth scanning is just as effective as manual inspection
in finding skimmers and is an order of magnitude faster.

To demonstrate how much money can be lost in a single day of undiscovered skimming, we begin in
Section~\ref{subsec:cardval} with an estimate of the value of that can be drawn from credit and debit cards by a
criminal. Then, in Section~\ref{sec:cardvol} we estimate the number of cards skimmed per day per skimmer.

In Section~\ref{sec:skimmersinwild}, we turn to two sources of data about skimmers recovered from the field: fuel
station inspection reports from the Arizona Department of Weights and Measures (AZWM) and statistics on skimmers from
the San Diego office of the San Diego office of the U.S. Secret Service (USSS).

\subsection{Economics of Carding}\label{subsec:cardval}

Stealing and monetizing stolen credit and debit card data, called \emph{carding} by its practitioners, is a well-studied
form of financial fraud. Here we summarize the public data available on the value attached to a stolen card by each of the
parties involved, namely the criminal, consumer, bank, and merchant used to cash out the card. Our results are summarized
in Table~\ref{tab:cardval}.

The valuation of a skimmer derives directly from the credit card data it collects. We can estimate this as:
%
\[W = \underbrace{\textrm{(card value per day)}}_{P}
\times \underbrace{\textrm{(cards per day)}}_{Q}
\times \underbrace{\textrm{(days deployed)}}_{D}\]
%
In the remainder of this section, we will estimate each of these quantities, $P$, $Q$, and $D$.

We find that a stolen card has a value of around \$600\footnote{If the fraud goes undetected, the cost is expected to
grow linearly with each additional day}, a skimmer captures around 20 cards per day, leading to a value to the criminal
per day of around \$12,000. Skimmers may be deployed for months without detection.

\paragraph{Criminal revenue} A carder can extract value from stolen card data in two ways: by selling the data to another
party, who will cash out the card, or by cashing it out himself. In order to quantify the impact of skimming, we will
focus primarily upon the ``cashing out'' stage, as this is the cost which is inevitably transferred to the victim,
merchant, and bank.

There are two widely discussed mechanisms of converting stolen card data to money: ATM withdrawals and product resale
\cite{krebsSkimReaper, viceInterviewWithCarder, arizonareport} For Debit cards and credit cards for which a PIN has
been set up may be cashed out by withdrawing money from an ATM. The other common use of skimmed cards is the purchase
of merchandise from an unwitting merchant for later resale.

The value a criminal can extract from a card depends on the type of card (debit or credit, high or low withdraw limit)
and what additional information is provided with the card (such as billing address). Online
forums~\cite{makingFirstMoney}, tutorials~\cite{cardingNewbieGuide, howToSucceedInStore}, and
interviews~\cite{viceInterviewWithCarder} have reported an expected value of \$400--800 per card, with
\$500 a common estimate. The value attained from each card also depends heavily on the merchant used and
the card issuer's fraud detection algorithm. If merchandise is purchased with the stolen card, it is usually resold
at a lower price. For example, for example, one forum post describes reselling Iphone X's for \$400, well below
today's list price of \$769~\cite{iphoneXSale}.

\begin{table}
    \begin{tabular}{l@{\quad}lrl}
    \toprule
    \multicolumn{2}{l}{\colname{Scheme}} & \colname{Value} & \colname{Reference} \\
    \midrule
    \multicolumn{2}{l}{\textbf{Black market price}} \\
    & Debit, no PIN & \$20--30 & \cite{meccadumps,sellcvv,dumpsto, dumpsPrtShip} \\
    & Debit with PIN & \$110--220 & \cite{legitshop, sellcvv, dumpsPrtShip} \\
    & Credit, no ZIP & \$10--25 & \cite{meccadumps,sellcvv,dumpsto, dumpsPrtShip} \\
    & Credit with ZIP & \$25--60 & \cite{meccadumps,sellcvv,dumpsto, dumpsPrtShip} \\
    \multicolumn{2}{l}{\textbf{Cash-out value}} \\
    & Credit or Debit (standard) & \$400--800 & \cite{makingFirstMoney, cardingNewbieGuide, howToSucceedInStore, viceInterviewWithCarder} \\
    & Credit (premium) & \$1,000 & \cite{cardingNewbieGuide, santandWithdraw, honeyMoneyTut}\\
   \multicolumn{2}{l}{\textbf{Bank and merchant loss}} \\
%   & DOJ & \$902 & \cite{harrell2017} \\
%   & Arizona W\&M & \$1,003 & \cite{arizonareport} \\
    & Credit & \$1,003 & \cite{arizonareport} \\
%   & ATM Skimming & \$650 & \cite{ATMIA} \\
    & Debit & \$650 & \cite{ATMIA} \\
    \multicolumn{2}{l}{\textbf{Consumer liability}} \\
    & Debit (> 60 days) & unlimited & 15~USC~1693g \\ % \cite[\S1693g]{15uscode} \\
    & Debit (< 60 days) & max \$500 & 15~USC~1693g \\ % \cite[\S1693g]{15uscode} \\
    & Debit (< 2 days) & max \$50 & 15~USC~1693g \\ % \cite[\S1693g]{15uscode} \\
    & Credit & max \$50 & 15~USC~1643 \\ % \cite[\S1643]{15uscode} \\
     \multicolumn{2}{l}{\textbf{Prosecuted loss}} \\
%   & USSC & \$500 & \cite{ussc-guidelines} \\
    & Credit or debit & \$500 & \cite{ussc-guidelines} \\
     \multicolumn{2}{l}{\textbf{Court documents}} \\
     & Credit & \$362--400 & \cite{hristov,cristea,alisuretove,mekhakian} \\
     & Debit & \$665--1132 & \cite{estrada,aqel} \\

    \bottomrule
\end{tabular}

    \caption{Stolen credit and debit card valuation. All prices in U.S. dollars.}
    \label{tab:cardval}
\end{table}

\paragraph{Consumer loss}
The U.S. has statutory limits on consumer liability for debit card fraud and credit card
fraud~\cite{15usc1643,15usc1693g}. Debit has a scaling liability depending on when the fraud is reported and
credit is limited to \$50 dollars regardless of delays in reporting. These numbers are summarized in
Table~\ref{tab:cardval}

\paragraph{Merchant and bank loss}
If consumer loss is capped by law, the loss must necessarily fall on the merchant involved (unwittingly) in the
cash out transaction or the issuing bank. In the case of a debit card cashed out through an ATM cash withdrawal, the
loss is borne by the bank. For credit card transactions, the apportionment of loss depends on a number of
factors~\cite{card-acceptance-guidelines-for-merchants}. Notably, with the introduction of chip (EMV) cards, Visa holds
a merchant who accepts a chip card at a chip enabled terminal not liable for fraud, but does hold a merchant who
accepts a chip card at a traditional magnetic stripe terminal liable~\cite{visa-liability-shift}.

Various estimates for loss associated with a stolen credit card have been proposed by the U.S. Sentencing Comission,
the U.S. Department of Justice, the ATM Industry Association, and the Arizona Weights and Measures. Conservatively, the
amount of damage caused by fraud is around \$500. Estimates from these sources are included in Table~\ref{tab:cardval}.
We have also seen skimming reports with this exact dollar amount~\cite{AZfiveHundred}.

%\note{There are a lot of numbers in cases available online that we should research for camera-ready. For example,
%in \emph{United States v. Alli} U.S. Dept. of Justice used maximum credit limit to estimate loss.
%\url{https://www.ussc.gov/sites/default/files/pdf/training/primers/2016_Primer_Loss.pdf} has a bunch of references
%to use as a starting point.}

\subsection{Daily Card Volume}
\label{sec:cardvol}



``A single compromised pump can capture data from roughly 30--100 cards per day.''~\cite{rippleshot}

\noteby{KL}{Need to revise rest of Section~\ref{sec:cardvol}.}

In order to quantify the damage caused by skimming, we will need to get an estimate of the number of unique cards a
skimmer sees each day.
%
Payment processor TSYS reported that in 2017 that 41\% of sales used a debit card, 34\% used a credit card, and 15\%
used cash \textit{at the pump}. \cite{TSYS}
%
Estimating based upon numbers given by the EIA, EPA, Dept. of Transportation, and a census study of gas station
infrastructure, this would mean an average of 86.49 (11.09) debit and and 71.72 (9.19) credit fuel sales per station
per day (per pump per day).
\footnote{Calculated from the average fuel sales of each gas station~\cite{NACS} in the united states,
average cost of a gallon of gas~\cite{EIAgasPrices}, average number of pumps~\cite{basker2017customer},
and the puchasing habits of the average consumer~\cite{EPAFuelEfficiency, averageDriveDeptOfTrans}.}
During their forensics, the USSS has also estimated 20-50 tracks per day per skimmer device, and as we will see in
the next section, we found this to be true in a case study of 10 skimmers whose memory we studied directly.
%
These numbers will vary depending on the environment, i.e. a more frequented station in a popular tourist destination
may see more cards.

The amount of track data stored on the skimmer at any given time is highly variable, however, depending on the behaviour
of the criminal.
%
Skimmers come in two varieties: ``daily'' and ``migratory''.
%
In ``daily'' types, data is collected and erased from the skimmer on a daily basis, most likely by a local criminal.
%
In ``migratory'' types, the criminal will install the skimmer in a position and then leave it installed for a couple
of months before coming back to collect the data, oftentimes driving long distances and hitting a wide, coastal area.


\subsubsection{Dumps of Deployed Skimmers}

In order to validate these initial hypotheses, we worked with government officials to attain information from skimmers
after forensic analysis had been run.
%
The memory of 10 gas station skimmers recovered from a single station revealed several descriptive statistics.
%
In total, there were 251 card reads recorded by the ten devices in tandem.
%
Duplicate and corrupted card swipes constituted 49 of the reads, leaving 202 unique card numbers.
%
All of the skimmers studied were of the ``daily'' variety, and results for the number of each type of card seen are
recorded in Table \ref{tab:skim-dump-res}.

While this data is biased by the location at which the skimmers were covered, it does give a rough idea of what the
average ``pull'' (visit to a station to get data) of a criminal will look like.
%
The station in question was ``average'' in customers per day;
%
it is likely that the values presented here could be over or underestimates depending on the location of installation.
%
Of the cards in question, only three explicitly appeared to be gas cards given out to fleet vehicles; the others could
be classified as personal cards.
%
Determining how many personal cards were gas cards was non-trivial given only knowledge of the BIN/IIN number.


\begin{table}
    \centering
    \begin{tabular}{ll}
    \toprule
    \textbf{Total Tracks} & 251 \\
    \textbf{Total Tracks Recovered} & 202 \\
    \midrule
    \colname{Type} & \colname{\# of Cards}  \\
    \midrule
    \textbf{Debit} \\
    \quad with PIN & 79 \\
    \quad with ZIP & 15 \\
    \quad without P/Z & 11 \\
    \textbf{Credit} \\
    \quad with ZIP & 71 \\
    \quad with PIN & 6 \\
    \quad without P/Z & 15 \\
    \textit{Unknown BIN/IIN} & 5 \\
    \midrule
    Avg. Tracks per Skimmer & 20.2 \\
    Std. Tracks per Skimmer & TODO \\
    Value per Skimmer & TODO? \\
    Total value & TODO? \\
    \bottomrule

\end{tabular}

    \caption{Summary of skimmer track statistics from ten recovered gas station skimmers.}
    \label{tab:skim-dump-res}
\end{table}

