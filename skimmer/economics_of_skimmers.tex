%!TEX root = paper.tex
\section{Economics of Skimmers in the Wild}
\label{sec:arizona}

From our discussion above, it is clear that the bill of materials for a typical skimmer falls some where in the
\$10--30 range, well under what a criminal might expect to extract from even a \emph{single} credit card.
Unfortunately, our visibility into the business of credit card skimming is limited: we are not aware of any reliable
public estimates of how much money a criminal makes from each skimmer, nor how each skimmer ends up costing merchants,
banks, and consumers. Nevertheless, we can estimate some of these quantities using public data, as well as a small
sample of skimmers recovered and made available to us.

The valuation of a skimmer---both revenue to the criminal and costs to the rest of us---derives directly from the
credit card data it collects. We can estimate this as:
%
\[W = \underbrace{\textrm{(card value)}}_{P}
\times \underbrace{\textrm{(cards per day)}}_{Q}
\times \underbrace{\textrm{(days deployed)}}_{D}\]
%
Thus, for the criminal, the revenue generated by a single skimmers is the product of expected monetized card value,
number of cards captured per day, and the number of days a skimmer is deployed. The consumer loss is, similarly, the
product of expected loss per card, the number of cards captured per day, and the number of days a skimmer is deployed.
In the remainder of this section, we will estimate each of these quantities, $P$, $Q$, and $D$. Of the three, $D$ is
the hardest to estimate, because, barring getting this information directly from the criminal, we can only report about
those skimmers that have been found in the wild, but not those that have been so well hidden as to remain undiscovered.
For the same reason, it is also difficult to know how many skimmers are currently operating in the wild. Nevertheless,
even studying only those skimmers that have been found, provides useful insights into the extent of the skimmer problem.

In Section~\ref{}, we turn to two sources of data about skimmers recovered from the field: fuel station inspection
reports from the Arizona Department of Weights and Measures (AZWM) and statistics on skimmers from the San Diego
office of the San Diego office of the U.S. Secret Service (USSS), the agency responsible for investigating financial
crimes, including skimmers, in the United States. \note{Add summary statement about what else we get from this data?}

We begin in Section~\ref{subsec:cardval} with an estimate of $P$ the value of credit and debit cards to all parties. Then,
in Section~\ref{sec:cardvol} we estimate the number of cards skimmed per day per skimmer. Finally, in
Section~\ref{sec:skimmersinwild}, we turn to an analysis of skimmers found in the wild.

\subsection{Economics of Carding}\label{subsec:cardval}

Stealing and monetizing stolen credit and debit card data, called \emph{carding} by its practitioners, is a well-studied
form of financial fraud. Here we summarize the public data available on the value attached to a stolen card by each of the
parties involved, namely the criminal, consumer, bank, and merchant used to cash out the card. Our results are summarized
in Table~\ref{tab:cardval}.

\paragraph{Criminal revenue} A carder can extract value from stolen card data in two ways: by selling the data to another
party, who will cash out the card, or by cashing it out himself. Debit cards are cashed out by withdrawing money from an
ATM or by using it as a credit card to make a purchase. The former requires the PIN associated with the card, while the
latter relies on signature-based verification like conventional credit cards. To cash out a credit card, the perpetrator
uses the credit card to make a purchase from an unwitting merchant, and then re-sells the merchandise. Credit card verification
varies by merchant and depends on the individual payment processing arrangement. Some merchants or points of sale may require
the cardholder to enter the five-digit billing ZIP code associated with a card. The value a criminal can extract from a
card depends on the type of card (debit or credit) and what additional information is provided with the card. For debit
cards, this additional information is the PIN, while credit cards may include the billing ZIP code, billing address, and
card security code (known as CVV2 on Visa and MasterCard). Because the subject of this paper is gas station skimmers, we
only distinguish between debit cards with and without a PIN, and credit cards with and without a ZIP code.

In our survey of sites and forums selling stolen credit and debit card data, we found that debit cards with a PIN sold for
\$110--120. Credits cards with a ZIP code sold for \$20--30, while credit cards with no additional information sold for
\$10--25~\cite{meccadumps,legitshop, sellcvv,dumpsto, dumpsPrtShip}. Business and premium credit cards, which
have higher credit limits, were listed for \$60.

Cashing out the card, rather than selling it some someone else, generates more revenue. Online forums~\cite{makingFirstMoney},
tutorials~\cite{cardingNewbieGuide, howToSucceedInStore}, and interviews~\cite{viceInterviewWithCarder} reported an
expected value of \$400--800 per card, with \$500 being the most common figure mentioned. Premium cards, with their
higher withdraw and daily purchase limits have a expected value two to three times this amount. Due to the nature of
this form of fraud, the value attained from each card by the criminal depends heavily on the merchant used and the card
issuer's fraud detection algorithm. Criminals will then re-sell the purchased merchandise, usually at a lower price.
For example, for example, one forum post describes reselling Iphone X's for \$400, well below today's list price of
\$769~\cite{iphoneXSale}.

\begin{table}
    \begin{tabular}{l@{\quad}lrl}
    \toprule
    \multicolumn{2}{l}{\colname{Scheme}} & \colname{Value} & \colname{Reference} \\
    \midrule
    \multicolumn{2}{l}{\textbf{Black market price}} \\
    & Debit, no PIN & \$20--30 & \cite{meccadumps,sellcvv,dumpsto, dumpsPrtShip} \\
    & Debit with PIN & \$110--220 & \cite{legitshop, sellcvv, dumpsPrtShip} \\
    & Credit, no ZIP & \$10--25 & \cite{meccadumps,sellcvv,dumpsto, dumpsPrtShip} \\
    & Credit with ZIP & \$25--60 & \cite{meccadumps,sellcvv,dumpsto, dumpsPrtShip} \\
    \multicolumn{2}{l}{\textbf{Cash-out value}} \\
    & Credit or Debit (standard) & \$400--800 & \cite{makingFirstMoney, cardingNewbieGuide, howToSucceedInStore, viceInterviewWithCarder} \\
    & Credit (premium) & \$1,000 & \cite{cardingNewbieGuide, santandWithdraw, honeyMoneyTut}\\
   \multicolumn{2}{l}{\textbf{Bank and merchant loss}} \\
%   & DOJ & \$902 & \cite{harrell2017} \\
%   & Arizona W\&M & \$1,003 & \cite{arizonareport} \\
    & Credit & \$1,003 & \cite{arizonareport} \\
%   & ATM Skimming & \$650 & \cite{ATMIA} \\
    & Debit & \$650 & \cite{ATMIA} \\
    \multicolumn{2}{l}{\textbf{Consumer liability}} \\
    & Debit (> 60 days) & unlimited & 15~USC~1693g \\ % \cite[\S1693g]{15uscode} \\
    & Debit (< 60 days) & max \$500 & 15~USC~1693g \\ % \cite[\S1693g]{15uscode} \\
    & Debit (< 2 days) & max \$50 & 15~USC~1693g \\ % \cite[\S1693g]{15uscode} \\
    & Credit & max \$50 & 15~USC~1643 \\ % \cite[\S1643]{15uscode} \\
     \multicolumn{2}{l}{\textbf{Prosecuted loss}} \\
%   & USSC & \$500 & \cite{ussc-guidelines} \\
    & Credit or debit & \$500 & \cite{ussc-guidelines} \\
     \multicolumn{2}{l}{\textbf{Court documents}} \\
     & Credit & \$362--400 & \cite{hristov,cristea,alisuretove,mekhakian} \\
     & Debit & \$665--1132 & \cite{estrada,aqel} \\

    \bottomrule
\end{tabular}

    \caption{Stolen credit and debit card valuation. All prices in U.S. dollars.}
    \label{tab:cardval}
\end{table}

\paragraph{Consumer loss}
The U.S. has statutory limits on consumer liability for debit card fraud and credit card
fraud~\cite{15usc1643,15usc1693g}. These limits report that debit card fraud has a scaling liability depending on
when the fraud is reported. They also report that credit card liability is limited to \$50 dollars regardless
of delays in reporting. Results are summarized in Table~\ref{tab:cardval}

\paragraph{Merchant and bank loss}
If consumer loss is capped by law, the loss must necessarily fall on the merchant involved (unwittingly) in the cash out
transaction or the issuing bank. In the case of a debit card cashed out through an ATM cash withdrawal, the loss is borne
by the bank. For credit card transactions, the apportionment of loss depends on a number of
factors~\cite{card-acceptance-guidelines-for-merchants}. Notably, with the introduction of chip (EMV) cards, Visa holds
a merchant who accepts a chip card at a chip enabled terminal not liable for fraud, but does hold a merchant who
accepts a chip card at a traditional magnetic stripe terminal liable~\cite{visa-liability-shift}.

In the case of money frauded through an ATM withdrawal, National Cash Register estimates that the bank suffers an
average loss of \$650 \cite{rippleshot}, and in cases of general credit card fraud, the Department of Justice has
estimated the direct loss to be \$902 dollars on average \cite{harrell2017}.

\paragraph{Other loss estimates}
The U.S. Sentencing Commission's Sentencing Guidelines assesses the loss associated with a stolen credit card at no less
than \$500~\cite[\S2B1.1]{ussc-guidelines}; this estimate is often used by the U.S. Department of Justice in the absence
of a more specific assessment. \note{There are a lot of numbers in cases available online that we should research for
camera-ready. For example, in \emph{United States v. Alli} U.S. Dept. of Justice used maximum credit limit to estimate
loss. \url{https://www.ussc.gov/sites/default/files/pdf/training/primers/2016_Primer_Loss.pdf} has a bunch of references
to use as a starting point.}

The Arizona Department of Weights and Measures places the ``direct loss per victim'' of skimming at
\$1,003~\cite{arizonareport}.


\subsection{Daily Card Volume}
\label{sec:cardvol}

``A single compromised pump can capture data from roughly 30--100 cards per day.''~\cite{rippleshot}

\noteby{KL}{Need to revise rest of Section~\ref{sec:cardvol}.}

In order to quantify the damage caused by skimming, we will need to get an estimate of the number of unique cards a
skimmer sees each day.
%
Payment processor TSYS reported that in 2017 that 41\% of sales used a debit card, 34\% used a credit card, and 15\%
used cash \textit{at the pump}. \cite{TSYS}
%
Estimating based upon numbers given by the EIA, EPA, Dept. of Transportation, and a census study of gas station
infrastructure, this would mean an average of 86.49 (11.09) debit and and 71.72 (9.19) credit fuel sales per station
per day (per pump per day).
\footnote{Calculated from the average fuel sales of each gas station~\cite{NACS} in the united states,
average cost of a gallon of gas~\cite{EIAgasPrices}, average number of pumps~\cite{basker2017customer},
and the puchasing habits of the average consumer~\cite{EPAFuelEfficiency, averageDriveDeptOfTrans}.}
During their forensics, the USSS has also estimated 20-50 tracks per day per skimmer device, and as we will see in
the next section, we found this to be true in a case study of 10 skimmers whose memory we studied directly.
%
These numbers will vary depending on the environment, i.e. a more frequented station in a popular tourist destination
may see more cards.

The amount of track data stored on the skimmer at any given time is highly variable, however, depending on the behaviour
of the criminal.
%
Skimmers come in two varieties: ``daily'' and ``migratory''.
%
In ``daily'' types, data is collected and erased from the skimmer on a daily basis, most likely by a local criminal.
%
In ``migratory'' types, the criminal will install the skimmer in a position and then leave it installed for a couple
of months before coming back to collect the data, oftentimes driving long distances and hitting a wide, coastal area.

\subsubsection{Dumps of Deployed Skimmers}

In order to validate these initial hypotheses, we worked with government officials to attain information from skimmers
after forensic analysis had been run.
%
The memory of 10 gas station skimmers recovered from a single station revealed several descriptive statistics.
%
In total, there were 251 card reads recorded by the ten devices in tandem.
%
Duplicate and corrupted card swipes constituted 49 of the reads, leaving 202 unique card numbers.
%
All of the skimmers studied were of the ``daily'' variety, and results for the number of each type of card seen are
recorded in Table \ref{tab:skim-dump-res}.

While this data is biased by the location at which the skimmers were covered, it does give a rough idea of what the
average ``pull'' (visit to a station to get data) of a criminal will look like.
%
The station in question was ``average'' in customers per day;
%
it is likely that the values presented here could be over or underestimates depending on the location of installation.
%
Of the cards in question, only three explicitly appeared to be gas cards given out to fleet vehicles; the others could
be classified as personal cards.
%
Determining how many personal cards were gas cards was non-trivial given only knowledge of the BIN/IIN number.


\begin{table}
    \centering
    \begin{tabular}{ll}
    \toprule
    \textbf{Total Tracks} & 251 \\
    \textbf{Total Tracks Recovered} & 202 \\
    \midrule
    \colname{Type} & \colname{\# of Cards}  \\
    \midrule
    \textbf{Debit} \\
    \quad with PIN & 79 \\
    \quad with ZIP & 15 \\
    \quad without P/Z & 11 \\
    \textbf{Credit} \\
    \quad with ZIP & 71 \\
    \quad with PIN & 6 \\
    \quad without P/Z & 15 \\
    \textit{Unknown BIN/IIN} & 5 \\
    \midrule
    Avg. Tracks per Skimmer & 20.2 \\
    Std. Tracks per Skimmer & TODO \\
    Value per Skimmer & TODO? \\
    Total value & TODO? \\
    \bottomrule

\end{tabular}

    \caption{Summary of skimmer track statistics from ten recovered gas station skimmers.}
    \label{tab:skim-dump-res}
\end{table}

