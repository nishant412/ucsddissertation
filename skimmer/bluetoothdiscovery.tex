\section{Bluetooth Discovery Process}
\label{sec:bluetoothdiscovery}

The Bluetooth pairing process is most likely familiar due
to its prevalence in consumer electronics. Delving into
greater technical detail, however, it follows the following
sequence.

The process begins with a stage called inquiry; this is the
primary discovery stage for bluetooth enabled worker devices by
a master device. The process is set up in a specific manner to
reduce conflict between device scanning, and speed up discovery
so that piconets with a master-worker dynamic can be quickly set
up. In inquiry cycles where the master device is not looking to connect
to devices for a specific purpose or service, the master device, known
as the inquirer, hops among 32 of Bluetooth's 79 1 MHz frequency channels,
according to a psuedo-random pattern seeded by a General Inquiry Access
Code (GAIC) defined within the standard. On each frequency, the device sends
out an initial inquiry packet consisting of the GAIC, a 28-bit CLK for
syncronization purposes, and a unique 48 bit address corresponding to the
master device. 625 microseconds after sending a pcket on frequency $x$, the
master device will listen on frequency $(x + 32) mod 79$ for a response from
a possible listening device. Meanwhile, the listening devices, known as scanners,
will listen given frequencies in 1.28 second ``scan windows'', before hopping
frequency in a predetermined manner in order to account for the pseudorandom
hopping of the inquirer device. Upon recieval of the initial inquiry packet from
the inquirer device, the scanner device will begin a backoff based upon a uniformly
distributed number of scan slots between 0 and 1,023, meaning between 0 and (roughly)
640 ms. Once it returns from this backoff state, the scanner device will begin
scanning again, and wait for a second inquiry packet from the master device. Once this
is recieved, the device will send out a Frequency Hopping Syncronization (FHS) packet,
in which the inquirer will learn the critical details of the device, such as
unique address. At this point, the inquirer device will finish its inquiry process, followed
by an initiation of paging with the worker device. It is during this initial paging
process that the bluetooth device name and other details will be exchanged. But notably,
device discovery itself takes around 10.24 seconds in official measurements, and only 5.12
seconds to discover 99\% of devices, according to more advanced analysis detailed in
[Peterson, Baldwin...].

In order to perform the bluetooth device discovery detailed within our studies of gas station
environments, we used the Android operating system's BluetoothAdapter interface to operating
system services. A BluetoothAdapter device discovery cycle typically consists of 12 seconds
of inquiry scanning followed by paging of those devices discovered in order to record device
names and details. There is a tradeoff present within following the default scan interval;
while allowing inquiry to complete ensures that various environmental factors
do not interfere with discovery and information is not lost, in a crowd-sourced application
where localization is a primary concern (in the case of skimmers, ensuring that the device
observed is within a pump), a larger number of datapoints for any given device is of larger
concern. Thus, we offered participants a variable setting for sane defaults between a 5
second scan and the android default scan time. By restricting the inquiry period to five
seconds, more datapoints were retrieved for each device. Additionally, because paging is handled
seperately by the system service, this still allowed us to retrieve device names. It is possible
that this led us to miss out on the discovery of some devices, and eager inquiry scanning to
increase the amount of localization points did interfere with paging, however, it roughly
doubled the number of geolocation points for each observed for each device, allowing for a
more fine-grained discovery of skimmers via the concentric ring based techniques discussed
in a later section.

