\section{Metric Justification}
\label{sec: metricjustification}

Considering our six metrics, there may be some disagreement on whether or not
such methods are capable of finding credit card skimmers, or have too high
of a false negative rate. The following section provides some justification
of the six metrics in order to best explain why they are effective in finding
skimmers. To address the potential criticism that these metrics ``may not be
enough'', we also address what a potential smart criminal may need to do to
circumvent our measures. This stated, from our experience in discussing with
both secret service and weights and measures staff during our tri-state study,
it is highly unlikely that any criminals are actively taking such actions, given
the skimmers which have thus far been found via ocular analysis of gas pump
ribbon wires. Additionally, the organizational structure of skimming
operations, as highlighted in earlier sections, leads to difficulty when a
criminal would like to implement these countermeasures. These organizational
roadblocks are noted when pertinent to the metric justification at hand.

\subsection{Filtering Out Classic Instead of BLE Devices}

Using the data collected of over 300 unique gas stations in a tri-state area, we
were able to qualitatively describe the Bluetooth environment of various locations
pervading day-to-day existence, including university cafeterias, research labs, and
most notably, gas stations. This allowed us to perform a latitudinal study which cataloged
what was normal or abnormal for various environments. It can be seen from a CDF of unique
classic vs ble MAC addresses observed at each of the gas stations that there is a much higher
probability of seeing at least one BLE device than there is of seeing a single classic device.
\todo{Ugh, try to get some actual statistics in here}
This suggests that classic devices are much more abnormal within the gasstation environment
than low energy devices. By flagging these devices as abnormal, we have a higher probability of
addressing one of these devices as being a skimmer.

Additionally, there is an incentive on the part of the criminal to build a skimmer using a
Classic instead of a BLE module. First, since the ribbon wire within the pump can provide
the skimmer with power, so energy usage is not a concern. Secondly, according to Bluetooth
SIG, BLE has an over-the-air transfer rate of around 0.25 Kbit/s, which Classic has a transfer
rate of up to 1Mbit/s. \todo{Citation Needed} Due to the desire to stay off of camera, and the avaliability of
Classic modules, there is more benefit to developing the skimmer with classic bluetooth. This
is confounded with the fact that there are no real trade-offs in using a Classic device, since
it's configurability is equivalent to a low energy module, unless criminals were deliberately
trying to circumvent our methods, or appear a normal part of the Bluetooth environment, which
does not currently seem to be a concern for them according to governmental sources.

\subsection{The Distinction between Unclassified and Classified Devices}

In the bluetooth communication protocol for both Classic and BLE devices, there is specification
allowing for a Bluetooth Device Class, used to identify to the user what the device might be
this is practical in most consumer applications, allowing users to determine whether the
device is a Bluetooth speaker vs. smart watch vs. any other Bluetooth electronic. It is also
used for interoperability in more complex systems. The Bluetooth device class is often
implemented as a stage of the consumer production process, and is not set by the manufacturer of
the chip itself. For freshly produced chips, it is typical to set the default response for
device class on the chip to be 0x1F00, which, according to specification \todo{citation},
indicates an unclassified device. The Android OS defaults to this value rather than null when
a response from an enquery scan is partially recieved \todo{citation, I think this is true.}.
Once unclassified modules enter production, such as in the ``Tile'' object-finding fobs, they
are given a device class, in this case 0x0000, for miscellaneous consumer electronic. Other
such classes include 0xYYYY, which is used for A/V handsfree devices. In figure X, we have
included a CDF of classic devices, near gas stations, with the Unclassified(0x1F00) and
Classified(any other two byte) device classes. The existence of an Unclassified device is
far more unlikely than that of a classic device than within a Classified class.

From the criminal perspective it requires far greater technical skill to modify the device
class for purposes of obsfucation, including modification of the chip's firmware and function,
and has no practical benefit for recovery, since this a superficial trait of device function.
It is unlikely that a criminal would take the measures neccessary to modify the device class
of the chip, and they typically do not, and thus this remains a valuable metric in finding
anomalous devices.

\subsection{A Hitlist}

The list of manufacturer-alotted MAC address ranges for the production of Bluetooth modules,
which we use to flag potential skimmers is, at first glance, one of the more questionable
metrics, and has the greatest chance of selection bias, of leading to false negatives.
However, there is some consideration which has gone into the selection of devices of devices
which are included on the hitlist. The initial hitlist has consisted of all those module
manuafacturers who readily, over the internet, have bulk amounts of Bluetooth Classic modules
for sale. This was arrived at through a scraping of SparkFun ... and other hardware
manufacturing websites. Additional entries can be added on the fly, in the case wherein a
skimmer is discovered using a new manufacturer which was previously unknown. When this happens,
our metrics will bring any missed entries up in ranking. It acts as an accompaniment to the
other metrics by reducing the range of manufacturers considered significantly.

There are some disagreements which can arise in regards to MAC randomization and obsfucation, as
a criminal could ostensibly perform these changes at a  low cost. While we have not yet been
informed of skimmers which have done so, this would also make credit card data recovery more
difficult for the criminal, as they would need to note what the original MAC address of the
device is, or flag it in some other way, using a name, or a pre-paired device to reconnect. Both
of these options have tradeoffs which any prospective criminal would need to consider. At the
current time such measures do not seem to be widely undertaken. In the event that criminals do
begin resorting to MAC address randomization, a different approach would need to be taken, one
such possibility is discussed later within this paper.

\subsection{Odd Name Clustering}

Aside from a hitlist of potentially dangerous devices, the next portion of the Bluetooth enquery
response packet which could be reasonably used to detect anomalous devices is the device name
field. In the case where the person retrieving the data is not the same as the person which
implanted the data, a device name can be an easily communicable feature, and reduce error during
the harvesting process. Additionally, the default names assigned by the manufacturers to
bluetooth chips commonly used in skimmers appear odd to the human observer, and stand out like a
sore thumb within the gas station environment. To illustrate this, imagine a day of seeing either
unnamed devices or devices named ``Michelle's IPhone'', and then seeing a device named
``RNBT-Q7XT'' (a common skimmer module name). The RNBT module immediately appears suspicious to
the observer, but it would be better to quantify this automatically. In order to accomplish
such a task, we implemented edit distance clustering based anomaly detection, based upon
the unsupervised nature of the data, as well as an intuition on the nature of device names
within the environment in order to better isolate those devices for which the default
manufacturer name was left unmodified.

For the distance metric, it was natural to use either hamming distance or Jaro Winkler, since
JW measures strings with the same prefix as more similar, the regular Levenshtein
edit distance metric would not account for this, and the bluetooth environment contains
many names that begin with the same string, e.g. ``Samsung TV...''.  Additionally
, Levenshtein would often confuse more complex device names like \"S02525b5df3834566R
\" and \"S277d67fe6263c208C\", because while these strings seem close to the human
eye, they are far in edit distance. Hamming distance, performed decently, since a
lot of the device names have the same letters in the same positions, but a manual
analysis of the dataset found Jaro Winkler to be more robust for this task.

There were three main approaches considered when clustering, we chose to do hierarchical
clustering, which made since, since it seemed that device names came in ``types''
. There are several possible approaches to this, One would be simply integrating
the nearest points into each-other in a cluster, or determining the centroid, and clustering
from that, but that which makes the most sense is ward's algorithm, which joins clusters to
create the minimum variance cluster. After clustering was performed, a somewhat arbitrary cutoff
point was defined, such that each cluster has no greater a cophenetic distance (
a measure of how closely the dendrogram preserves the distances between the
original points) than 1.0. If the points were in a cluster with fewer than fifty other names,
they were considered odd , which tended, from visual analysis, to have a pretty high false
positive rate, which was preferred in the case of this metric, since a well defined,
generally applicable model for the environment would be quite complex and the remaining
set of points was small enough to be visually verified by the experimenters. \todo{This
  doesn't seem scientific enough?} Of course, such a model can always be improved, and we
continue to do so as time progresses.

Of course, criminals have several courses of action when responding to such analysis; they could 
leave the device unnamed, or name it something inconspicuous, along the lines of ``Tile''. In
practice, this does not seem to be the case. Criminals often name the devices, according to
government officials, along the lines of ``Police Scanner 13'', which is classified as odd under
our metrics. Additionally, while a single metric may not hit, as the earlier analysis suggests,
several others have a high likelihood of doing so, and the device, as we have seen in practice,
will remain flagged. Ignoring this metric, despite the potential flaws, would be a mistake.

\subsection{Near Gasstation}

The feature of being near a gas station should make implicit sense. Non-superficial skimmers can,
in fact, be installed in locations other than gas pumps, such as parking pay stations, but
this work has focused on gas pumps as these are a prime target for skimmer installation, and
serve indicative of the general environment of skimmers in the wild. The techniques for finding
skimmers in gas pumps can certainly be applied more generally, but with the scope of this
study, being near a gasstation is a significant flag for an interesting device. Given our
crowd-sourced approach to detection, some method of limiting the scope of observation was
neccessary. One could imagine applying such scoping when addressing the issue of parking pay
station skimmers. If we did not limit the geolocation scope of skimmer observation to areas in
which skimmers could reliably be discovered, it may have been impossible to sort out the amount
of noise, and to develop accurate measurements of the environment.

\subsection{Seeing Double (Seen Twice)}

This final metric is perhaps the simplest. A skimmer is a static, unmoving device, but our
observer base is not. Additionally, the environment of a gas station has a high degree of flux;
individuals with bluetooth enabled devices come to the pumps and leave them constantly. Thus,
if a device is seen twice at the same location, it is a important flag; this device is ruled out
from being transient, and has a much higher probability of being a skimmer when paired with the
other metrics presented here. The difficulty of determining this metric when a criminal
implements MAC randomization is addressed in a later section. For this metric, we chose to flag
devices which were seen in the same location with greater than an hour between measurements.

