\section{The Nature of Skimmers}
\label{sec:skimmernature}
Card Skimmers and Shimmers (those which read from the chip
rather than the mag stipe) have been a blossoming field of
research in recent time. Previous research has been done
to detect the superficial variety of skimmers, by detecting
a double read of the mag stripe as it is used within a machine.
[1]. The forms of skimmers have advanced more rapidly than research
has alluded, however, and varieties have evolved for the purpose of
infecting potential environments.  Bluetooth and GSM enabled chips are
now readily built and deployed, as was noted in our discussions with
police and government officials in states which have high automotive
usage[2]. Bluetooth based skimmers, in particular, have become prevalent
in states such as Arizona, due to cheaply available hardware and ease
of connection with consumer electronics [3]; these skimmers, which
typically are inaccessible by physical retrieval and interface only
through wireless interfaces are referred to as ``deep-insert''.

In the gas pump environment, these skimmers are attached onto the ribbon wire
between the mag reader on the pump and the POS system which mediates the
transaction. The device typically consists of a microcontroller, embedded
memory, and a bluetooth transciever or GSM chip. The device draws power from
the pump itself, and stores information which it intercepts as it travels
across the wire. At an undetermined point in time, a client desiring to
retrieve data from the device may connect to it via bluetooth, establish
an RFCOMM serial data transfer socket, and pull data from memory. In this
fashion, criminals are able to collect credit card data from various victims
by installing the device in a frequently-used gas pump, download it at a
later time, and then make use of this data to perform illicit purchases.

There are two downsides to such an approach: difficulty in installation 
and difficulty of retrieval. Installation requires access to the internals
of the device being skimmed itself. In the case of gas pumps, this difficulty
is variable depending upon the pump type. For older Gilbarco Brand pumps, for
instance which constituted roughly 31\% of one metropolitan area in which we
surveyed, there exists a single master key, and the ribbon wire to which the
device must be attached is accessible in the frontmost compartment accessible
by the key. Newer pumps are not as accessible, and thus require either more
expensive hardware such that a deep insert skimmer may be installed
directly via the card slot or through a time intesive process. Retrieval
requires the use of a specialized application which can interface to the
device; in the case of GSM, such an interface might be tracable if the
skimmer is discovered, and in the case of Bluetooth, someone must visit the
pump for the data to be retrieved. Bluetooth, in particular, is vulnerable to
honeypotting, a mitigation strategy which we have performed and which is
discussed in a later section.

The obvious benefit on behalf of do-gooders to a device using Bluetooth is
the existence of an interface by which the device may be discovered, however,
doing so is non-trivial, and is the focus of this paper. Detection of these
malicious devices involves hunting for abnormal signal strength patterns,
geofencing, fingerprinting of inquiry scan response data, and crowdsourced
data colleciton.