%!TEX root = paper.tex
\section{Related Work}
\label{sec:relatedwork}
\paragraph{Skimmer Detection and Prevention}

In recent work, Scaife et al. surveyed gas pump skimmer detection and
prevention mechanisms~\cite{scaifeoakland}.
%
They found that several popular Bluetooth-based skimmer detection applications
use a only MAC prefix or device name matching.
%
The results of our study show how the Bluetooth profile of skimmers in these
applications can be improved to detect more skimmers, and to flag fewer
legitimate devices as skimmers.
%
We also find that Bluetooth-based scanning is an effective way to augment
manual gas pump inspections.
%
Scaife et al. also introduced SkimReaper~\cite{skimreaper2018}, an effective
tool for detecting external skimmers.
%
SkimReaper is a credit-card shaped device that an official can swipe in a card
reader to detect if the reader has an additional read head: indicating that the
reader has an external skimmer attached to it.
%
%This tool detects the presence of a second read head capturing card data in a
%card reader.
%
%Physical contact makes it possible to build a circuit which reveals the
%presence of a second read head.
%
However, SkimReaper can not detect internal skimmers because they do not
add an additional read head.
%
Additionally, the PCI Security Standards Council have released guidelines for
preventing external skimming~\cite{skimmingprevention}.
%
Criminals may start using Bluetooth to retrieve card data
from external skimmer. If they do, we
demonstrate that Bluetooth scanning can augment
these existing external skimmer detection and prevention methods.
%
%Wireless scanning makes the discovery internal skimmers possible without
%manual inspection.
%
%
%Bluetana presents an intermediate stopgap to internal skimming; however, more
%work on preventative measures is needed.

\paragraph{Bluetooth}

Prior work has evaluated the effectiveness of Bluetooth scanning for detecting
and localizing Bluetooth devices.
%
They found that Bluetooth signal strength (measured by an Android smartphone)
is effective for localizing Bluetooth
devices~\cite{liu2014face,wang2013bluetooth}.
%
This work inspired us to use signal strength to detect if a Bluetooth device
appears to be installed inside of a gas pump.
%
Previous studies also examined how long it takes to detect a Bluetooth
device from stationary observers and moving vehicles.
%
They found that Bluetooth devices are often detected in less time than the
Bluetooth standard suggests~\cite{murphy2002using,
peterson2006bluetooth,haartsen1998bluetooth}.
%
This work supports our findings that skimmers are often discovered within the
first few seconds of passing by a gas station.
%
%It also demonstrates the usefulness of drive-by scanning in Bluetooth skimmer
%discovery.

\paragraph{Inventory Attacks}

Prior work has demonstrated that user privacy can be violated by inspecting the
characteristics of a user's device~\cite{ziegeldorf2014privacy}.
%
These so called \emph{inventory attacks} have been demonstrated for Bluetooth
Low-Energy, RFID, and even web browsers~\cite{van201050,
vastel18fp,fawaz2016protecting}.
%
Our work demonstrates a Bluetooth-based inventory attack against malicious
devices, can be used to protect the privacy of consumers.

%\paragraph{Payment Processing Security} Credit card spoofing and transaction
%fraud is a common area of research; while criminals currently rely on the
%trivial vulnerabilities of magnetic strips \cite{magspoof}, it is conceivable
%that newer systems (such as Chip-and-PIN) will prevent magnetic strip based
%attacks. Card issuers feel that removing sensitive data from the magnetic strip
%on cards will help to solve the problem \cite{pcidss}. However, newer literature
%has demonstrated attacks on chip, NFC, and tokenized payment systems
%\cite{bar2005known} \cite{bond2014chip} \cite{roland2013cloning}
%\cite{bai2017picking}. In response to the growing popularity of chip-based
%systems, criminals have begun to develop shimmers: chip-based deep-insert
%Bluetooth-enabled skimmers powered by the EMV reader itself
%\cite{krebshimmer}. The nature of shimmers (Bluetooth) indicates that wireless
%data exfiltration will continue to be an issue. Criminals will always be
%motivated to integrate wireless exfiltration mechanisms as physical retrieval
%involves a risk of being caught. It follows that work on wireless-based
%detection mechanisms like Bluetana will continue to be relevant as vendors
%transition to newer payment mechanisms.  \footnote{This discussion disregards
%the use of vendor-specific payment apps, such as those which have been devised
%by Shell \cite{shellapp}, as these network-based mechanisms come with a host of
%other problems which are outside the scope of this paper.}
