%!TEX root = paper.tex
%\section{Introduction}
\label{sec:intro}

The ubiquity of wireless access links has made it easier for attackers to attack public infrastructure. Today, criminals are implanting illicit wireless links to gain covert unauthorized access to gas pumps~\cite{arizona-2018,florida-2018}.
%
These illicit wireless links are "hidden in the noise" of tens of others such links at public locations.
%
In this chapter, I describe the results of a 19-month metropolitan scale field measurement study to understand the feasibility of using link-layer information from smartphone Bluetooth scans to identify these illicit links.
%
In particular, this field study was aimed at defending against a type of illicit link --- Bluetooth-based payment card skimmers.
%

% The story is can we augment slow and methodical with fast and maybe less good
% to save lots of money. Officers, NOT consumers.

% === Big picture picture ===
Payment card skimming attacks at gas pumps have reached alarming levels.
%
In 2018, law enforcement officials recovered 972 skimmers from gas pumps in Florida~\cite{florida-2018} and 148 skimmers from Arizona~\cite{arizona-2018} alone.
% 524 stations %at 11 stations %
Based on industry estimates, a single skimmer can capture 30--100 credit cards per day~\cite{rippleshot} and each card, based on estimates from law enforcement officials, nets the criminal an estimated \$500~\cite{ussc-guidelines}, resulting in a daily loss of \$15,000--50,000 per day of operation for each skimmer.\footnote{In Section~\ref{sec:background:money}, we compare these quoted estimates to other sources, and find them to be in agreement.} Less is known about how long a skimmer remains in operation, but allowing for even one day of operation per skimmer, 2018 losses exceed \$16 million across these two states.

Gas pumps are an ideal skimming target.
%
Gas pumps have relatively weak security: their payment circuitry can be accessed with universal keys or crowbars, and reading payment data is as easy as tapping into a ribbon cable (Section~\ref{sec:bkgd-skimhw}).
% 
Gas pump skimmers can be hidden inside of a gas pump enclosure, making them difficult to detect.
% 
As a result, inspectors have resorted to manually opening the pumps to inspect their wiring for skimmers.
%\footnote{These inspections are infrequently triggered by complaints.}
% 
Gas pump skimming has become so pervasive that the Arizona Department of Agriculture, Weights and Measures Division (AZWMSD) now checks for skimmers while doing routine inspections.\footnote{For example,
the ``Vapor Recovery Inspection Pre-Test Checklist'' has a checkbox for ``Checked for Skimmers''.}
%
From 2016 to 2018, the AZWMSD looked for skimmers in \azskiminspect~gas station inspections.
%
Inspectors found skimmers in only \azpercentskimfound~of these inspections.

Unfortunately, Law Enforcement (LE) rarely catch criminals while they are collecting payment data from gas pump skimmers.
%
The reason is, many gas pump skimmers are equipped with Bluetooth connectivity~\cite{krebs-siphoning,krebs-pos,krebs-gang,krebs-mexico}.
%
This allows criminals to remain in their car while wirelessly retrieving card payment data.
%
While Bluetooth is a vital tool for criminals to exfiltrate data from gas
pumps, it also could be an opportunity to make it easier to detect skimmers.

% === Our solution: Study Bluetooth ==== 

%that allow for operation from a safe distance
In this measurement study, we evaluate the effectiveness of using Bluetooth scans from a smartphone to detect these payment card skimmers.
%
Indeed, if a skimmer can be detected with 
a smartphone, then authorities can discover and remove
skimmers passively and quickly while they visit a gas station for
other reasons.
% 
We built a smartphone application to perform this study, called Bluetana.
%
Bluetana collects all Bluetooth scan data that is available via
the Android Bluetooth APIs.
%---including information that has not been considered
%in the few Bluetooth skimmer detection tools that are available for consumers.
%
We equipped \numvolunteers~volunteers in six U.S. states with smartphones
running Bluetana. 
%
Our volunteers have collected wireless scans at \visitedgasstations~gas stations, where
%
they observed a total of \totalbtobserved~Bluetooth devices.
%
In these scans, Bluetana detected a total of \totalskimmers~skimmers
installed at gas stations in Arizona, California, Nevada, and Maryland, and it
%
was the sole source of information
that led law enforcement to find \totalskimmersBluetana~skimmers.
%
%Assuming
%all of these skimmers were operating for at least one day, 
%we estimate that Bluetana prevented \Bluetanafraudprevented~in fraud.
%
%with Bluetana was the trigger for after we detected each skimmer we informed
%law enforcement, and they removed the skimmers and collected them as evidence.
%
%We estimate that the skimmers discovered by during this study stopped a
%total of \Bluetanafraudprevented~in fraud per day they were operating.

%
% They are also one of the largest owner-operated infrastructure that is slow to
% update to secure technology: there are $\sim$135,000 gas stations across the
% U.S., many of which still use magnetic stripe card readers and insecure PIN
% pads~\cite{emv2020}.
%
%This provides the key benefit of detecting skimmers before they have been used
%for enough fraud to trigger authorities to perform manual inspections (if it
%ever does).
%

%Bluetooth-based detection provides the additional benefit of revealing
%well-hidden skimmers.

% === Results ===
Our study is the first comprehensive look at how 
skimmers can appear in Bluetooth scans.
%
Namely, we observe that it is feasible to differentiate skimmers from other
common Bluetooth devices that appear in Bluetooth scans at gas stations (e.g.,
vehicle telemetry collectors).
%
Using a combination of Bluetooth scan link-layer information fields such as Class-of-device, MAC address and Device name, we were able to uniquely identify skimmers at gas stations.
%
%The main differentiating factor for the skimmers we observed, is that the
%Bluetooth Class-of-Device---a parameter not collected by any consumer Bluetooth
%scanning applications that we are aware of---is ``Uncategorized''.
%for all of
%the skimmers we observed \noteby{JC}{for all of the skimmers}.
%
We also find that signal strength is a reliable way to determine if a Bluetooth
device is located near a gas pump, and thus could be a skimmer.
%
%even if it does not match the characteristics of typical skimmers.
%
%We also inadvertently ran a natural experiment where we observed the duration
%that skimmers can operate undetected in gas pumps: at least six months.

%(2) we observe that Bluetooth scanning is an effective tool for finding skimmers,

%for
%the Bluetooth signals from skimmers is today; and how effective it
%will be in the future.
 


Our study reveals several problems with consumer Bluetooth-based skimmer
detection applications~\cite{scaifeoakland,sparkfunapp,skimplus}:
%
(1) there are many legitimate products that appear at gas stations that use the
same Bluetooth modules as known skimmers; therefore, MAC address-prefix based detection 
may lead to false positives,
%
(2) there are many Bluetooth modules used in skimmers that do not comply with IEEE MAC
assignment requirements.
%
We also debunk advice on how to find skimmers with Bluetooth scans from authorities~\cite{ag-mn-skimmers} and viral information from social media~\cite{snopesskimmers}.
%
For instance, none of the skimmers we found using Bluetooth scans 
have a name that is a long string of letters and numbers.

Performing this in-depth study brought to light several important operational
lessons learned about the importance of detecting skimmers with Bluetooth.
%for
%improving skimmer inspections.
%
Using Bluetooth scans, officials detected skimmers while
driving by gas stations that they otherwise would not have inspected.
%
We also witnessed several instances where an inspector tried to find a skimmer,
but could not find it on their first pass looking inside a gas pump. However they persisted and found it based on the
knowledge that a suspected skimmer had appeared in
Bluetooth scans.
%
Surprisingly, we observed that there are skimmers installed in the same gas station, or
city, that have very similar MAC addresses---indicating their source is a
single criminal entity.
%
We even found skimmers installed hundreds of miles away that had surprisingly
close MAC addresses.

The rest of the chapter is organized as follows: Section~2 provides background
on internal gas pump skimming: their construction, monetary incentive, and prevalence in
the wild. Section~3 is an overview of our large-scale Bluetooth scan collection methodology.
%for large-scale Bluetooth skimmer measurement study.
In Section~4, we
present the results of our study: what the skimmers we detected look like, how they compare to skimmers recovered independently by Law Enforcement, and
whether they are well hidden in the Bluetooth environment. In Section~5, we present
possible counter measures to the Bluetooth detection.  In Section~6 we present the operational lessons we
learned about skimming and criminal investigation procedure, while performing
our large scale measurement study. Section~7 is related work, and we conclude in
Section~8.

\if 0 % extra text {{{

Internal skimmers passively siphon track data, PIN keystrokes, and power, from
the PoS terminal's circuitry that sends all of this data in plain-text inside
the device (Section~\ref{sec:background:design}).

%
The primary advantage of Bluetooth-based detection is it can be crowdsourced:
authorities do not have to open PoS terminals to find skimmers, they can be
detected by anyone with a smartphone that is near a compromised PoS terminal.
%

We provide the first estimate of how much money criminals can make from
skimming (\todo{XXX} per skimmer) through a detailed analysis of carding
forums, combined with our own analysis of actual credit card data stored on
skimmers that were retrieved by law enforcement.
%


Also unlike ATMs, gas dispensers have been slow to update to new secure
technology because they are individually owned and operated\footnote{ATMs are
operated by a few large companies (i.e., Diebold and Cardtronics).}.

, and criminals delete the credit card numbers off of the skimmers as they
download them. 

The quintessential
skimmer is a a false face-plate overlay attached to an ATM's card slot that
records the card's track data, and a pinhole camera aimed at the PIN keypad to
record the PIN entry. 

ATM skimmers alone are estimated to cost over \$2 billion
globally~\cite{cnbc-skimmers}.

Upon finding a skimmer in a terminal during a manual inspection, they installed a
camera inside it to capture images of the criminal in the act of
retrieving the data~\cite{camera-in-pump}.

As evidence of this, the payment card industry has pushed back the deadline to
require gas stations to convert EMV chip payment technology to
2020, and it is likely to be pushed back again.



\fi %}}}
