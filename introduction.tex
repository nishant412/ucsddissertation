Urban society relies upon the safe and secure operation of wireless communication links that are part of all modern electronic systems. From personal devices like smartphones, smartwatches, laptops to public infrastructure like grid equipment, lighting, fuel and transportation systems, wireless ad-hoc links (e.g. Bluetooth and WiFi) have given users within wireless range a capability to remotely monitor and access the electronic equipment safely and conveniently. 

However, this convenience comes at a cost as these wireless links expose our modern electronic systems to a whole new set of security and privacy attacks; these can have disastrous consequences to our public health and safety, economy and even national security. 
Unfortunately, we lack the empirical data to audit these wireless links and understand the security threats, and even lack effective tools to collect this empirical data for doing comprehensive wireless audits in the first place. Today we have millions of these wireless ad-hoc links in deployment, but don’t have visibility into their security and privacy problems. Even worse, for a lot of cases we don’t even know what wireless links are depl

Unfortunately, we lack the toolkits, datasets and even empirical methods necessary to comprehensively audit and secure this important “network” of non-Internet connected wireless links. Indeed, the design of effective security methodology is complicated by the diverse (in manufacturers, end uses, physical hardware variations) and distributed (over large geographic area) nature of this “network”.

There are several examples today that reveal the tension between convenience of wireless links and their security. Continuously transmitting BLE beacons allow our smartphones to effectively locate other personal devices like other smartphones, smartwatches, laptops [cite,cite] both indoors and outdoors. This feature enabled convenient smartphone based contact tracing, and was crucial to public health and safety during the recent COVID-19 crisis. Unfortunately, this feature also converts our personal devices into homing beacons, causing an invasion of personal privacy. Attackers can covertly and passively listen to these beacons and use the link and physical layer features to uniquely identify and track our location, making it a convenient tool for stalkers.

These wireless links also provide a safe and efficient way for operators to access urban infrastructure equipment. For example, a maintenance worker need not spend time physically accessing a circuit breaker on a dangerous high voltage line; they can remotely connect over Bluetooth and run diagnostics at a safe distance [cite]. Today, there are thousands of such links deployed in infrastructure equipment such as traffic lights, speed signs, reclosers, transformers, fuel equipment all across the US, protecting urban operators from hazardous situations. However, this wireless link can be misused by an attacker to covertly gain unauthorized access to the infrastructure equipment[cite circuit breaker, traffic light], allowing them to change configurations causing outages or damage to the equipment. Even worse, attackers may even bypass physical security on these equipment and retrofit their own wireless links to gain illegitimate access. For instance, criminals have been installing Bluetooth based skimmers in payment card terminals to commit billions of dollars in fraud each year [cite].

To protect this “network” of non-Internet connected wireless links against attackers, we need to design tools and datasets to build an empirical understanding of potential attack surfaces, and then secure them using defenses. To build the empirical knowledge, we need to perform urban scale measurement campaigns using wireless scanning across this entire “network” of ad-hoc links. These wireless auditing campaigns utilize the information obtained from device discovery scanning to enumerate wireless links, using either fixed or mobile scanners. These scans let us enumerate what wireless links are present in different electronic systems and what they are in use for. This consequently lets us develop an empirical understanding of how and what legitimate links can be attacked, and what illegitimate links are being implanted by attackers. Indeed, even attackers run these wireless scans to identify a “target” wireless link, and therefore an empirical measurement campaign truly gives us an insight into what weak links the attackers can even identify and choose to target.

In this dissertation, I explore the key research question to be addressed for performing these wireless scanning based measurement – identification. In particular,

Put it another way, these wireless links are obfuscated or hidden among the tens of wireless
Our success at auditing using wireless scanning really 
How accurately can we identify (distinguish) a particular target wireless link in real-world environments using wireless scans, especially in the presence of several other wireless links? Answering this question requires us to develop tools to find these wireless links, and also develop an empirical understanding of the real-world limitations

when there are several An empirical understanding not only helps us design better auditing tools, but also helps us understand the practical limitations

The key question to answer with regards to a wireless audit is really whether we can use wireless scans to find a particular wireless link to target. This indeed is a hard problem to answer, because in any real world scenario there are tens or even hundreds of wireless links at a time, and therefore any one wireless link is effectively hidden or obfuscated in the noise. It really de

\section{Challenges}
Existing techniques are insufficient to audit the security and privacy threats of these wireless ad-hoc links, because of their inherent distributed and diverse nature. Indeed, these properties also complicate the design of new empirical methods and tools to achieve our goals of accurate and fast identification of wireless links.

The heterogeneity of these links makes it untenable to audit these links by simple vulnerability analysis of individual wireless devices[cite] that are used throughout an urban area. These links are extremely diverse – they are used in a wide variety of end applications (from personal devices to public infrastructure) across urban areas. In addition, there are a large number of vendors manufacturing the chipsets/devices used in these wireless links. Even for the same manufacturer, there are several different modules/chipsets  and even physical hardware variations within the same module (e.g. manufacturing variations in the clock crystal). Consequently, there are potentially thousands of different manufacturers and models of deployed ad-hoc links; a number of these are undocumented, and for a number of these we don’t even have the relevant user manuals to even begin to test the wireless device/chipset.

Any particular wireless device in the real world is hidden or obfuscated in the “noise” of several other wireless links around. There is minimal information available in wireless scans from device discovery packets. Wireless scanning uses information present in device discovery packets (e.g. inquiry request/response, advertisement beacons) to perform identification of wireless links. This process only provides us basic information like MAC addresses, a human-friendly name, type of device at the link layer. Identifying individual wireless links using just this information is like trying to find a needle in a haystack – there are millions of wireless links across personal devices and public infrastructure. At any real world location, we observe tens to hundreds of such wireless links making it difficult to pinpoint any one link. For example, 80\% of gas stations we observed had 15 different Bluetooth devices. Furthermore,  the exact same make and model of wireless device may be used in a diverse set of applications at the same location. For instance, we observed that at the same gas station, the exact same Bluetooth module was used in a speed sign, a traffic light as well as for an illegitimate skimmer at the gas station. Indeed this “hidden in the noise” problem has even led researchers to believe  that wireless scanning is insufficient to identify “hidden” wireless links [scaife oakland].

Even wireless hardware properties are ineffective at identifying or unobfuscating these ad-hoc links. We can measure wireless signal properties(e.g. CFO, CSI) from the received wireless scan packets. These wireless signal properties are due to manufacturing variations in the wireless hardware and therefore represent a particular

The distributed nature of these non-Internet connected links makes it impossible to audit them using existing wireless device fingerprinting techniques [cite Danny, Louis]. These links are geographically spread throughout cities and nations, and therefore there isn’t a central vantage point to observe the links. Wireless scanning through wardriving can be used to perform  measurement campaigns throughout large urban areas. Unfortunately, existing wardriving techniques are insufficient because wireless scanning is too slow to comprehensively discover all wireless links across entire metropolitan areas. The process of wireless scanning can take several tens of seconds to enumerate all wireless links nearby; this is too slow when scanning across urban areas in a moving vehicle and leads to missed devices, which can even be misconstrued as evidence of a device actually being hidden or obfuscated in the noise.





and perhaps some basic hardware signal properties like CFO, IQ offset at the physical layer.

hardware variations across two same devices very similar so hard to differentiate



we dont get the spatial context which can only be gotten by doing wardriving

Existing techniques for wireless device auditing,


\section{Thesis}
\section{Insights}
\section{Contributions}
