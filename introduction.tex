\begin{dissertationintroduction}
    Big Picture

    Over the years, Bluetooth has become a popular short range communication technology in virtually everything - from personal devices (phone, smartwatch, tracker), payment systems, proximity detection systems, transportation (fleet trackers, car alarms) to industry and infrastructure (power grid, speed signs, traffic lights). These Bluetooth radios are fairly critical to not only our individual lives, but even to the proper functioning of our cities as a whole. 
    
    Improper or unauthorized use of any of these devices can have effects from the scale of individual safety to failure of financial and administrative institutions. Unfortunately, the deployment and maintainers of such electronic infrastructure make assumptions that they are aware of where and how such wireless radios are being used, and that Bluetooth technology protections will protect their privacy.
    
    In today’s world, Bluetooth sensors beacons are deployed in various urban places to enable proximity sensing, targeted ads and asset tracking and interaction with users. Further on, with the widespread use of social contact tracing, these systems have become even more important to ensuring spread of the pandemic
    
    In various kinds of infrastructure, Bluetooth modules that can be scanned for and connected to are being fitted as wireless implants. The aim is to allow maintainers and users easy access to internal buses for data and command exchange with the electronic systems, without needing to physically access the hardware. The maintainer simply needs to scan for the Bluetooth module and connect to it to run the necessary diagnostics. In transportation, Bluetooth modules are used with the internal OBD ports to allow easy vehicle diagnostics for applications such as vehicle repair and fleet tracking using a simple smart device. Further on, equipment such as traffic light controllers, speed signs are fitted with similar Bluetooth modules. On the power grid, equipment such as transformers, cap bank controllers etc which are located high up on poles are fitted with a Bluetooth radio for maintenance crews to diagnose issues and change settings without needing to climb the pole. All of these use cases show how Bluetooth devices have become integral to the smooth functioning of our urban areas.
    
    Unfortunately these electronic systems fitted with wireless radios work under the assumption of proper utilization of the radios, and inherent protections offered by Bluetooth technology. These assumptions don’t hold true, making them vulnerable to an attacker. Primarily, these devices are generally kept in discoverable or scannable mode, to make them visible and usable for legitimate users. This simple action makes them easily visible to any and all including the attacker.
    
    Bluetooth beacons rely on techniques such as MAC randomization to prevent unauthorized tracking, which has been proven to have several vulnerabilities. This can result in severe issues to user data privacy
    
    In case of infrastructure, these systems are typically deployed as install-and-forget, and since any person accessing such systems needs to be in close proximity the assumption is only legitimate users will attempt to access the system. This is thought to be a stronger guarantee especially since these systems are widespread, its hard to even know that such systems exist unless in proximity, and therefore need not be secured or monitored. To enable any maintenance personnel access to these systems, often they are left unsecured or use hardcoded passwords available in manuals online. This creates potential attack surfaces, even if one of which is exploited can cause huge problems, ex., one misconfigured traffic light controller at an intersection can cause massive accidents.
    
    In the most egregious cases, attackers can misuse lack of physical security and monitoring of infrastructure by installing their own implants to continuously exfiltrate data and often times misconfigure such infrastructure. A common and widespread example is criminals placing skimmers in gas pumps to steal payment card data of consumers. THis results in massive financial losses year on year to our financial institutions. The irony in this situation is that the same implicit assumption of invisibility unless in proximity is what criminals utilize knowing that no one will be able to find their device.
    
    Researchers in security understand the various problems associated with insecure Bluetooth radios. However, we don't have an insight into the scale of the problem in an urban setting. Particularly, what are the various Bluetooth devices that are being used, in what type of end equipment, and how they are being used. The fact that a number of such devices are visible to the world, can aid us in our search for these. Identification of the variety of Bluetooth devices in an urban setting can reveal potential attack surfaces, and is a necessary first step in securing our critical infrastructure.
    
    Challenges
    The task of identifying what Bluetooth devices are in deployment and attributing them to a defined end equipment is extremely challenging. We know that the devices we are hoping to observe are visible in Bluetooth scans, and so logic would dictate we just scan for them when in proximity. While  this approach sounds reasonable in theory, it’s extremely difficult to scale at urban levels. Such scans must be done in close proximity to devices due to the short range of Bluetooth radios. Wardriving based data collection of Bluetooth scans across the city is possible, but to cover all types of devices we need a lot of drivers.. Additionally, since wardrivers don’t stop to collect multiple scans around a piece of end-equipment (to begin with they don’t even know what to stop for), we end up getting very few scan records per device, if any.
    
    Even if such a wardriving exercise for Bluetooth scanning is performed, and a dataset is collected, identification is difficult because of limited information. Bluetooth scan responses (for classic Bluetooth) and advertisement scans (for Bluetooth LE) contain very little in terms of identifying information even for the radio. Bluetooth is intended as a short range peer to peer communication link, and therefore a user has no difficulty in identifying their Bluetooth device even with a simple name or MAC address. However, Bluetooth is so popular in all types of personal and commercial equipment, that any given location a simple scan leads to hundreds of different devices. Even if there are particular infrastructure in a small area that we are interested in, it still is a needle in a haystack problem. Further on, there is massive usage of popular Bluetooth radios, and particularly with implants, end-equipment manufacturers don’t bother changing the Bluetooth parameters to reflect information about the end-equipment. For example, during a wardriving exercise at a particular gas station,  I observed 4 radios with a very similar name. One was a speed sign, one was a fleet tracking system, one was a gas pump skimmer, and one was a traffic light controller.
    
    At the PHY layer, previous work indicates that by capturing the raw signals of WiFi radios, we can extract hardware impairment based information that is a unique identifier of the radio. We can use low cost SDRs to capture raw signals of the device discovery packets. Unfortunately, such techniques are hard to perform for Bluetooth radios. Bluetooth uses a narrowband GFSK modulation scheme with extremely short packets in device discovery. Measuring the hardware impairment based parameters with high resolution is extremely difficult. Furthermore, these parameters are known to be sensitive to the wireless environment and may not provide a unique enough identifier.
    
    Bluetooth devices can provide a list of services/UUIDs for devices that connect to them. A natural inclination is therefore to attempt connecting to some of these devices, especially ones that are unsecured or use default passwords. However, this is a wrong practice, especially when dealing with critical infrastructure in the field. We don’t have prior knowledge of command structure, and therefore we may inadvertently trigger an action that can lead to failure. In the case of illegal implants, connecting is even worse because we can overwrite critical forensic information that law enforcement uses to catch the criminals (e.x. last connected MAC address). Therefore, we have to restrict ourselves to simply the unconnected device discovery scan information for identification.
    
    In summary, to perform Bluetooth device identification at an urban scale we must rely solely on the limited information contained in device discovery packets collected through wardriving efforts, making it a challenging task.
    
    Thesis statement 
    In this dissertation, my goal is to design a generalized mechanism for identifying how Bluetooth devices are utilized in an urban environment. I will utilize information contained only in Bluetooth device discovery packets at both MAC and PHY layers, and combine it with geographic context information obtained from a data scanner. This information will be obtained from a combination of public and personally collected multi-year wardriving datasets. Within these constraints, I will show that even with a large number of Bluetooth devices in the environment,  it is possible to identify which devices correspond to urban infrastructure, and also that persistent PHY identification is possible even with MAC layer identifiers randomizing.
    
    My aim is to defend the following thesis statement: In the presence of a large number of Bluetooth devices at urban scale, we can utilize Bluetooth wardriving scans to 1) Identify particular equipment of urban infrastructure in which the legitimate devices are being used, 2) uniquely identify anomalous or illegitimate wireless devices in infrastructure, 3) Obtain persistent PHY layer identifiers that uniquely identify a wireless radio and are robust to changes in wireless environments
    
    
    Inventory or identification is a key first step in security audits for ensuring security of networked devices. Using this inventory, we can identify legitimate devices and the potential attack surfaces they expose, but also illegitimate devices interacting with our network. Such an identification is an important part of securing our organizational networks including critical infrastructure, and has been identified as an important initial step in various security frameworks like NIST, CISA (full names here)
    
    In recent times, a new class of connectivity solutions have emerged in urban infrastructure - Short range peer-to-peer devices. These devices, predominantly Bluetooth and WiFi modules, are attached to internal buses or available ports on infrastructure equipment. They provide easy, localized wireless access to maintainers for running diagnostics and configuring these infrastructure equipment. Authorized users simply scan for these wireless modules and connect to them using a smartphone/tablet, and are able to gain access to the functionality available.
    
    Current security audit methods are able to inventory network connected devices over any geographic scale, or unconnected devices within the confines of an organization building. Unfortunately, these short range peer-to-peer wireless links are at the intersection : they are not connected to a network, and are spread over entire metropolitan areas, and therefore have limited visibility with existing inventory approaches. This has resulted in several open attack surfaces for nefarious elements. Not only that, this limited visibility has encouraged criminals to attach their own such short range modules to infrastructure to gain unauthorized access (ex. gas pump skimmers)
    
    In this dissertation I tackle the problem of metropolitan scale identification of short range peer-to-peer wireless devices in infrastructure, by utilizing device discovery data collected from wardriving efforts. Using prior work, I show that it is possible to identify illegal wireless devices in fuel infrastructure using limited device discovery information, and that raw signal information can be utilized to extract persistent identifiers in large scale field settings. Finally, I propose a project to perform large scale inventory of such short range peer-to-peer wireless devices in urban infrastructure across multiple counties in the US. I discuss the technical and logistical challenges associated with such an effort, proposed ideas to tackle those challenges, and open research questions that I hope to answer.
    
\end{dissertationintroduction}